\documentclass[a4paper]{article}

\usepackage[T1]{fontenc}
\usepackage{ProfLycee}
\usepackage{math-vh}

% *** Réglage des headers ***
\fancyhead[L]{TMATHS4 --- Lycée Victor Hugo}
\fancyhead[R]{O. Laurent --- Anneée 2024-2025}
% *** Réglage des headers ***

\begin{document}

\begin{center}
  {\scshape\LARGE Fonction ln - Test\par}
  \vspace{0.5cm}
\end{center}

\NomPrenom{}


\begin{enumerate}
    \item Simplifier les expressions suivantes :
    \begin{itemize}
        \item $A = \ln(e^2)$
        \zonereponse{1cm}
        \item $B = \ln(5)+ln(6)$
        \zonereponse{1cm}
        \item $C = \dfrac{1}{2} \ln(3)$
        \zonereponse{1cm}
    \end{itemize}
    \item Déterminer l'ensemble de définition de la fonction $f$ définie par $f(x) = \ln(x^2+6x+10)$
        \zonereponse{6cm}
    \item Résoudre l'équation:
    \[ \ln(3x-4) = \ln(x^2-4) \]
        \zonereponse{9cm}
\end{enumerate}


\end{document}
        