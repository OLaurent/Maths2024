\documentclass[a4paper]{article}

\usepackage[T1]{fontenc}
\usepackage[french]{babel}
\usepackage{ProfLycee}
\useproflyclib{ecritures}
\usepackage{math-vh}
\usepackage{tabularx}
\usepackage{yhmath}
\usepackage{pst-tree,pst-plot,pst-text,pst-func,pst-math,pst-bspline,pstricks-add,pst-arrow}
\newcommand{\vect}[1]{\overrightarrow{\,\mathstrut#1\,}}
% *** Réglage des headers ***
\fancyhead[L]{TMATHS4 --- Lycée Victor Hugo}
\fancyhead[R]{O. Laurent --- Année 2024--2025}
% *** Réglage des headers ***

\begin{document}

\begin{center}
  {\scshape\LARGE Chapitre 9 --- Fonction logarithme népérien\par}
\end{center}

\begin{activite}{}{} 
On considère la fonction exponentielle définie par $f(x)=e^x$.

\begin{enumerate}
  \item Justifier que l'équation $f(x)=2$ admet une unique solution $\alpha$ dans l'intervalle $[0;10]$. Donner une valeur approchée de $k$ à $10^{-1}$ près.
  \item Soit $k$ un nombre réel. Conjecturer le nombre de solutions de l'équation $e^x=k$ en fonction de $k$.
  \item Résoudre $e^x=1$.
\end{enumerate}
\end{activite}

\begin{activite}{}{}
On considère maintenant la représentation graphique de la fonction exponentielle, donnée ci-dessous:
\begin{center}
  \begin{tikzpicture}[x=1cm,y=1cm,%unités
  xmin=-3,xmax=6,xgrille=1,xgrilles=1, %axe Ox
  ymin=-3,ymax=6,ygrille=1,ygrilles=1] %axe Oy
      \GrilleTikz
      \AxexTikz[Police=\small]{-3,-2,-1,1,2,3,4,5,6}
      \AxeyTikz[Police=\small]{-2,-1,1,2,...,6}
      \AxesTikz
      \CourbeTikz[very thick,blue,samples=250]{(exp(\x)}{-3:1.8}
      \node[blue] at (2.5,5.5) {$\mathcal{C}_f$};
  \end{tikzpicture}
\end{center}

\begin{enumerate}
  \item Tracer sur le graphique la droite $(d)$ d'équation $y=x$, puis tracer la courbe symétrique de $\mathcal{C}_f$ par rapport à $(d)$.
  \item On note $ln$ la fonction appelée "Logarithme népérien" correspondant à la représentation graphique obtenue. Tracer cette fonction à la calculatrice.
  \item Déterminer graphiquement l'ensemble de définition de cette fonction.
  \vspace{2cm}
  \item Déterminer graphiquement le sens de variation de la fonction $ln$.
  \vspace{2.5cm}
  \item Déterminer graphiquement le signe de la fonction $ln$.
  \vspace{2.5cm}
\end{enumerate}
\end{activite}

\pagebreak
\begin{activite}{}{}
  Le baron écossais John Napier (ou Neper), théologien et activiste protestant issu d'une grande famille écossaise, s'intéressait aux arts mathématiques.

  Dans son livre, écrit en latin et publié en 1614,
  \textquote{Mirifici logarithmorum canonis descriptio}
  (La Description de la règle merveilleuse des logarithmes),
  Napier annonce qu'il a trouvé un moyen étonnant de simplifier,
  non seulement les calculs trigonométriques, mais aussi tous les calculs qui font perdre tant de temps
  à ceux qui pratiquent les mathématiques : longues multiplications et divisions, extraction de racines carrées et cubiques.
  
  Dans sa préface, il insiste sur le fait qu'il a dû y consacrer beaucoup de réflexion
  mais que le résultat est tel qu'il peut affirmer qu'il n'existe pas de meilleure méthode. 



  \underline{Procédé calculatoire:} : on cherche à construire des tables numériques à deux colonnes, mettant en
  correspondance les nombres de telle manière qu'à la multiplication de deux nombres de la colonne de
  gauche corresponde l'addition de deux nombres de la colonne de droite.

  \begin{center}


  \begin{tabular}{cc}
    \begin{minipage}{6cm}
      \begin{tabular}{|C{2cm}|C{3cm}|}
        \hline
      0,1 & \\
      \hline
      0,5 & \\
      \hline
      1 & \\
      \hline
      1,5 & \\
      \hline
      2 & $0,69315$ \\
      \hline
      3 & $1,09861$ \\
      \hline
      4 & $1,38629$ \\ 
      \hline
      5 & $1,60944$ \\
      \hline
      6 & $1,79176$ \\
      \hline
      7 & $1,94591$ \\
      \hline
      8 & $2,07944$ \\
      \hline
      9 & $2,19722$ \\
      \hline
      10 & $2,30259$ \\
      \hline
      11 & $2,39790$ \\
      \hline
      12 & $2,48491$ \\
      \hline
      13 & $2,56495$ \\
      \hline
      14 &  \\
      \hline
      \end{tabular}
    \end{minipage}&
    \begin{minipage}{6cm}
      \includegraphics[width=6cm]{img/ac3.png}
    \end{minipage}
  \end{tabular}
\end{center}


 Voici un exemple illustrant la méthode: on a $2\times 3 = 6$ et $0,69315+1,09861=1,79176$.

 On vérifie que le nombre qui correspond à $6$ dans la colonne de droite est $1,79176$.

 \begin{enumerate}
  \item Quel nombre doit-on écrire en face de $1$? Quel nombre doit on écrire en face de $14$?
  \vspace{1cm}
  \item Quand on divise deux nombres de la colonne de gauche, que peut-on dire de ceux de la colonne de droite?
  \vspace{1cm}

  En déduire les nombres à écrire en face de: $1,5$; $0,5$ et $0,1$.

  \item Pour désigner les nombres de la colonne de droite, on invente le mot \textquote{logarithme}, forgé à partir des deux
  mots grecs logos (rapport) et arithmos (nombre entier naturel). En effet si les nombres de gauche sont dans
  un rapport constant (c'est à dire en progression géométrique), alors ceux de droite sont à différence constante
  (c'est à dire en progression arithmétique). Vérifier cette propriété en considérant dans la première colonne les
  nombres 1,2,4,8.
  \vspace{2cm}

  \item Sans calculatrice, déterminer le logarithme népérien de $\dfrac{495}{7}$.
  \vspace{2.5cm}
 \end{enumerate}
\end{activite}



\pagebreak

\section{Fonction logarithme népérien}

\begin{definition}{}{}
\begin{itemize}[label=\textbullet]
  \item Pour tout réel $a>0$ l'équation $e^x=a$, d'inconnue $x$, admet une unique solution dans $\R$.

  Cette solution se note $x=ln(a)$ et s'appelle le logarithme népérien de $a$.

  \item La fonction qui, à tout réel $a>0$, associe le réel $ln(a)$ s'appelle la \textbf{fonction logarithme népérien}.

  C'est la \textbf{fonction réciproque} de la fonction exponentielle. Elle est définie sur $]0;+\infty[$.
\end{itemize}
\end{definition}

\begin{propriete}{}{}
  Dans un repère orthonormé, les courbes représentatives des fonctions exponentielle et le logarithme népérien 
  sont symétriques par rapport à la droite d'équation $y=x$.

  \begin{center}
    \begin{tikzpicture}[x=1cm,y=1cm,%unités
    xmin=-3,xmax=6,xgrille=1,xgrilles=1, %axe Ox
    ymin=-3,ymax=6,ygrille=1,ygrilles=1] %axe Oy
        \GrilleTikz
        \AxexTikz[Police=\small]{-3,-2,-1,1,2,3,4,5,6}
        \AxeyTikz[Police=\small]{-2,-1,1,2,...,6}
        \AxesTikz
        \CourbeTikz[very thick,blue,samples=250]{(exp(\x)}{-3:1.8}
        \CourbeTikz[very thick,red,samples=250]{(ln(\x)}{0.05:6}
        \CourbeTikz[very thick,ForestGreen,samples=250]{\x}{-3:6}
        \node[blue] at (2.5,5.5) {$y=e^x$};
        \node[red] at (5.5,2.5) {$y=ln(x)$};
        \node[ForestGreen, below] at (5.5,4.9) {$y=x$};
    \end{tikzpicture}
  \end{center}

  
\end{propriete}

\begin{propriete}{}{}
\begin{itemize}[label=\textbullet]
  \item Pour tout réel $b>0$ et pour tout réel $a$, $e^a=b \iff a=ln(b)$.
  \item $ln(1)=0$ et $ln(e)=1$.
  \item Pour tout réel $a>0$, $e^{ln(a)}=a$
  \item Pour tout réel $a$, $ln(e^a)=a$.
  \item Pour tous réels $a$ et $b$ positifs, $a=b \iff ln(a)=ln(b)$
  \item Pour tous réels $a$ et $b$ positifs, $a<b \iff ln(a)<ln(b)$. 
\end{itemize}
\end{propriete}

\begin{methode}{Résoudre une équation ou inéquation avec logarithme}{}
Résoudre sur $]0;+\infty[$ les équations ou inéquations suivantes:

\begin{tabular}{c|c}
  \begin{minipage}{8cm}
    \begin{center}
      
      $3ln(x)+2=11$
      \vspace{4cm}
    \end{center}
  \end{minipage}&
  \begin{minipage}{8cm}
    \begin{center}
      
  $ln(x)<ln(2)$
      \vspace{4cm}
    \end{center}
  \end{minipage}
\end{tabular}
\end{methode}

  
\begin{exercices}{}{}
  32--33 p.244, 34--37 p.245, 40--41 p.245
  \end{exercices}


\begin{methode}{Résoudre une équation ou inéquation avec logarithme - Cas général}{}
  Résoudre sur les équations ou inéquations suivantes:
  
  \begin{tabular}{c|c}
    \begin{minipage}{8cm}
      \begin{center}
        
        $ln(5x+1)=ln(x)$
        \vspace{10cm}
      \end{center}
    \end{minipage}&
    \begin{minipage}{8cm}
      \begin{center}
        
    $ln(4x^2-x)\leqslant ln(3x)$
        \vspace{10cm}
      \end{center}
    \end{minipage}
  \end{tabular}
  \end{methode}

\begin{exercices}{}{}
57--74 p.246
\end{exercices}

\pagebreak

\section{Relation fonctionnelle et propriétés algébriques}
\begin{theoreme}{}{}
  Pour tous réels $a$ et $b$ strictement positifs, $ln(a\times b)=ln(a)+ln(b)$.
\end{theoreme}
\begin{example}{}{}

\begin{itemize}[label=\textbullet]
  \item $ln(35)=ln(7\times5)=ln(7)+ln(5)$
  \item $ln(\dfrac{1}{5})+ln(10)=ln(\dfrac{1}{5}\times 10 )=ln(2)$
\end{itemize}
\end{example}


\begin{propriete}{}{}
\begin{itemize}[label=\textbullet]
  \item Pour tous entier $n$ et $a$ réel strictement positif, $ln(a^n)=n\times ln(a)$.
  \item Pour tout réel $b$ strictement positif, $ln\left(\dfrac{1}{b}\right)=-ln(b)$.
  \item Pour tous réels $a$ et $b$ strictement positifs, $ln\left(\dfrac{a}{b}\right)=ln(a)-ln(b)$.
  \item Pour tous réels $a$ strictement positif, $ln\left(\sqrt{a}\right)=\dfrac{1}{2}ln(a)$. 
\end{itemize}
\end{propriete}

\begin{example}{}{}
  \begin{tabularx}{.95\linewidth}{X X}
  \textbf{1)} $ln(3^4)=4ln(3)$ & \textbf{2)} $ln\left(\dfrac{1}{2}\right)=-ln(2)$\\
  \textbf{3)} $ln\left(\dfrac{5}{3}\right)=ln(5)-ln(3)$ & \textbf{4)} $ln\left(\sqrt{7}\right)=\dfrac{1}{2}ln(7)$\\
  \end{tabularx}
\end{example}

\begin{methode}{Résoudre une équation en utilisant les relations fonctionnelles}{}
  Résoudre l'équation  $ln(x-4)+ln(x-2)=ln(3)$.

  \vspace{4cm}
  

  \end{methode}

\begin{exercices}{}{}
43--48 p.245, 95 p.248, 96--111 p.249
\end{exercices}
\begin{methode}{Résoudre une équation avec inconnue en exposant}{}
  Résoudre l'équation  $5^n \leqslant 10~000$.

  \vspace{4cm}
  

  \end{methode}

\begin{exercices}{}{}
116-117 p.249
\end{exercices}


\pagebreak
\section{Dérivabilité et limites de la fonction $ln$}

\subsection{Dérivabilité}

\begin{propriete}{}{}

\begin{itemize}[label=\textbullet]
  \item 
  La fonction logarithme népérien est continue et dérivable sur $]0;+\infty[$ et pour tout $x>0$:
  $$ln'(x)=\dfrac{1}{x}$$
  \item Soit $u$ une fonction dérivable sur un intervalle $I$ telle que, pour tout $x\in I$, $u(x)>0$. Alors la fonction $ln \circ u$ est dérivable sur 
  $I$ et $(ln \circ u)'=\dfrac{u'}{u}$.
\end{itemize}
\end{propriete}

\begin{demonstrationp}{}{}
  \vspace{8cm}
\end{demonstrationp}

\begin{example}{}{}
Soit $f$ la fonction définie par $f(x)=ln(x^2+5)$, qui est de la forme $ln(u(x))$ avec $u(x)=x^2+5$.

On a alors $f'(x)=\dfrac{2x}{x^2+5}$.
\end{example}

\begin{exercices}{}{}
27--28 p.244
\end{exercices}


\begin{methode}{Calculer une fonction dérivée avec logarithme}{}
Soit $f$ la fonction définie sur $\R$ par $f(x)=ln(x^4-x^2+1)$. Calculer $f'(x)$.

\vspace{6cm}
\end{methode}

\begin{exercices}{}{}
75--88 p.247, 89--94 p.248
\end{exercices}

\pagebreak
\subsection{Limites}
\begin{propriete}{}{}

  \begin{itemize}[label=\textbullet]
    \item  $\lim\limits_{x \to 0^+} ln(x)=-\infty$
    \item $\lim\limits_{x \to +\infty} ln(x)=+\infty$
  \end{itemize}
\end{propriete}
\begin{propriete}{Croissances comparées}{}
  Pour tout entier $n\geqslant 1$:
  \begin{itemize}[label=\textbullet]
    \item  $\lim\limits_{x \to 0^+} x^n ln(x)=0$
    \item $\lim\limits_{x \to +\infty} \dfrac{ln(x)}x^n{}=0$
  \end{itemize}
\end{propriete}

\begin{demonstrationp}{ $\lim\limits_{x \to 0^+} x ln(x)=0$}{}
  \vspace{10cm}
  
\end{demonstrationp}

\begin{exercices}{}{}
50--56 p.245, 122--128 p.250, 29--139 p.251
\end{exercices}





  

\end{document}