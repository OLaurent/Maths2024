\documentclass[a4paper]{article}

\usepackage[T1]{fontenc}
\usepackage[french]{babel}

\usepackage{ProfLycee}
\usepackage{math-vh}
\usepackage{tabularx}
\usepackage{eurosym}
\usepackage{pst-plot,pst-tree,pstricks,pst-node,pst-text}
\usepackage{FenetreCas}

% *** Réglage des headers ***
\fancyhead[L]{TMATHS4/TMATHS2 --- Lycée Victor Hugo}
\fancyhead[R]{20 janvier 2024}
% *** Réglage des headers ***

\newcommand{\E}{\mathbb{E}}
\newcommand{\vect}[1]{\overrightarrow{\,\mathstrut#1\,}}

\begin{document}

\begin{center}
  {\scshape\LARGE Devoir surveillé \No4\par}
  \vspace{0.5cm}
\end{center}

\NomPrenom{}
\medskip

\begin{exerciceinterro}{\hspace{2cm}/8 points}{}
\textbf{Partie A} :

\medskip

Soit $g$ la fonction définie sur $[0;+\infty[$ par $g(x)=(2-x)e^x+1$.

\begin{enumerate}
	\item \textit{Donner la valeur exacte de $g(0)$ et $g(1)$.}
	
	On a $g(0)=(2-0)e^0+1=2+1=3$ et $g(1)=(2-1)e^1+1=1+e$.
	



	\item \textit{Calculer $\lim\limits_{x \to +\infty} g(x)$}.
	
	On a $\lim\limits_{x \to +\infty} 2-x=-\infty$ par somme donc $\lim\limits_{x \to +\infty} (2-x)e^x=-\infty$ par produit 
puis:
$$\lim\limits_{x \to +\infty} g(x)=-\infty$$




	\item \begin{enumerate}
		\item \textit{Calculer $g'(x)$ pour tout $x$ dans $[0;+\infty[$.}
		
		Pour tout $x\in[0;+\infty[$:

		$$g'(x)=-1\times e^x+(2-x)e^x=(1-x)e^x$$
		




		\item \textit{En déduire le tableau de variations de la fonction $g$ }.
	
		On sait d'après la question précédente que pour tout $x\in[0;+\infty[$$g'(x)$ est du signe de $1-x$, donc le tableau de variations de $g$ est:

		\begin{center}
		\begin{tikzpicture}
			\tkzTabInit{$x$ / 1 , $x-1$ / 1, $g'(x)$ / 1, $g(x)$ / 1.5}{$0$, $1$, $+\infty$}
			\tkzTabLine{, +, z, -, }
			\tkzTabLine{, +, z, -, }
			\tkzTabVar{-/ $3$ , +/ $1+e$, -/ $-\infty$}
		 \end{tikzpicture}
		\end{center}
		
	\end{enumerate}	
	\item \begin{enumerate}
		\item \textit{Démontrer que l'équation $g(x)=0$ admet une unique solution $\alpha$ sur $[0;+\infty[$.}

		D'après le tableau de variation: pour tout $x\in[0;1]$, $g(x)>0$, donc l'équation $g(x)=0$ n'admet pas de solution sur l'intervalle $[0;1]$.

		\begin{itemize}
			\item La fonction $g$ est continue comme produit et somme de fonctions continues.
			\item  $g$ est stricement décroissante sur $[1;+\infty[$.
			\item $g(1)=1+e$ et $\lim\limits_{x \to +\infty} g(x)=-\infty$, donc $0\in]-\infty;1+e]$.
		\end{itemize}

		D'après le théorème des valeurs intermédiaires, l'équation $g(x)=0$ admet une unique solution sur $[1;+\infty[$, donc sur $[0;+\infty[$ d'après ce qui précède.
		
		\item \textit{Compléter le programme Python suivant pour que l'appel $encadrement()$ renvoie les deux bornes d'un 
		encadrement de $\alpha$ d'amplitude inférieure à $0,01$.}
\begin{center}
\begin{CodePythonLst}*[Largeur=8cm]{}
def f(x):
    return (2-x)*exp(x)+1

def encadrement():
    a=0,b=3	
    while (b-a) > 0.01 :
        m=(a+b)/2
        if f(a)*f(m) > 0:
            a = m
        else:
            b = m
    return a,b
\end{CodePythonLst}
\end{center}

\item \textit{\`A l'aide de la calculatrice, déterminer un encadrement d'amplitude $10^{-2}$ de $\alpha$.}

On obtient à la calculatrice $2,12 \leqslant \alpha \leqslant 2,13$.

\item \textit{Donner, sans justification, le tableau de signe de $g(x)$.}

$g$ s'annule en $\alpha$ donc 

\begin{center}
	\begin{tikzpicture}
		\tkzTabInit{$x$ / 1 , $g(x)$ / 1}{$0$, $\alpha$, $+\infty$}
		\tkzTabLine{, +, z, -, }
	\end{tikzpicture}
	\end{center}


	\end{enumerate}
\end{enumerate}

\medskip
\textbf{Partie B} :

Soit $f$ la fonction définie et dérivable sur $[0;+\infty[$ par $f(x)=\dfrac{x-1}{e^x+1}$.

Un logiciel de calcul formel nous donne les résultats suivants, que l'on peut utiliser sans justification:
          
\begin{CalculFormelXcas}[TexteOptions=XCas Calcul Formel, Sep=false]
	\LigneCalculsXcas{deriver($(x-1)/(e^x+1)$)}{$\dfrac{(2-x)e^x+1}{(e^x+1)^2}$}
\end{CalculFormelXcas}


\begin{enumerate}
	\item \textit{\`A l'aide de la partie $A$, déterminer les variations de la fonction $f$ sur $[0;+\infty[$.}

	On a d'après le logiciel de calcul formel, pour tout $x\in[0;+\infty[$ : $f'(x)=\dfrac{g(x)}{(e^x+1)^2}$

	$f'$ est donc du signe de $g$, donc on en déduit donc le tableau de variations de $f$ 


	\begin{center}
		\begin{tikzpicture}
			\tkzTabInit{$x$ / 1 , $g(x)$ / 1, $f'(x)$ / 1, $f(x)$ / 1.5}{$0$, $\alpha$, $+\infty$}
			\tkzTabLine{, +, z, -, }
			\tkzTabLine{, +, z, -, }
			\tkzTabVar{-/ $-\dfrac{1}{2} $ , +/ $ $, -/ $0$}
		 \end{tikzpicture}
		\end{center}



	\item \textit{Montrer que $e^\alpha=\dfrac{1}{\alpha-2}$ puis en déduire que $f(\alpha)=\alpha-2$.}
	
	On a:  $g(\alpha)=0 \iff (2-\alpha)e^\alpha+1=0 \iff (2-\alpha)e^\alpha=-1 \iff e^\alpha = -\dfrac{1}{2-\alpha} \iff e^\alpha=\dfrac{1}{\alpha-2}$.

	On en déduit que $f(\alpha)=\dfrac{\alpha- 1}{e^\alpha+1}=\dfrac{\alpha-1}{\dfrac{1}{\alpha-2}+1}=\dfrac{\alpha-1}{\dfrac{1+\alpha-2}{\alpha-2}}=\dfrac{\alpha-1}{\dfrac{\alpha-1}{\alpha-2}}=(\alpha-1)\times \dfrac{\alpha-2}{\alpha-1}=\alpha-2$.
	\item \textit{En déduire un encadrement d'amplitude $10^{-2}$ de $f(\alpha)$.}
	
	D'après la Partie A: $2,12 \leqslant \alpha \leqslant 2,13 \iff 0,12 \leqslant \alpha-2 \leqslant 0,13 \iff0,12 \leqslant f(\alpha)\leqslant 0,13$.

\end{enumerate}

\medskip

\end{exerciceinterro}

\pagebreak

\begin{exerciceinterro}{\hspace{2cm}/4 points}{}



\medskip 

Soit $f$ la fonction définie sur l'intervalle $[-3~;~4]$ par :

\[f(x) = \dfrac{\text{e}^x}{1 + x^2}\]

On note $\mathcal{C}_f$ sa courbe représentative dans un repère orthogonal.

\medskip

\begin{enumerate}
\item 
	\begin{enumerate}
		\item \textit{Déterminer la dérivée de la fonction $f$ sur l'intervalle $[-3~;~4]$.}
		
		$f$ est dérivable sur $[-3~;~4]$ et pour tout $x\in[-3~;~4]$:
		
		$f'(x)=\dfrac{e^x(1+x^2-e^x\times 2x)}{(1+x^2)^2}=\dfrac{(1-2x+x^2)e^x}{(1+x^2)^2}=\dfrac{(1-x)^2e^x}{(1+x^2)^2}$.
		\item \textit{Justifier que la courbe $\mathcal{C}_f$ admet une tangente horizontale au point d'abscisse 1.}
		
		On a $f'(1)=\dfrac{(1-1)^2e}{(1+1^2)^2}=0$, donc la courbe $\mathcal{C}_f$ admet une tangente horizontale au point d'abscisse $1$.

	\end{enumerate}
\item  Les concepteurs d'un toboggan utilisent la courbe $\mathcal{C}_f$ comme profil d'un toboggan. Ils estiment que le toboggan assure de bonnes sensations si le profil possède au moins deux points d'inflexion.

\medskip

\begin{minipage}{0.48\linewidth}
\psset{unit=0.75cm}
\begin{pspicture*}(-4,-1.95)(4.25,4.1)
\psgrid[gridlabels=0pt,subgriddiv=1,gridwidth=0.15pt](-4,0)(4,4)
\psaxes[linewidth=1.25pt,labelFontSize=\scriptstyle](0,0)(-4,-0)(4.25,4.1)
\psplot[plotpoints=2000,linewidth=1.25pt,linecolor=blue]{-3}{4}{2.71828 x exp x dup mul 1 add div}
\rput(0,-1){Représentation de la courbe $\mathcal{C}_f$}
\uput[ul](3,2){\blue \small $\mathcal{C}_f$}
\end{pspicture*}
\end{minipage} \hfill
\begin{minipage}{0.48\linewidth}
\psset{unit=0.75cm}
\begin{pspicture}(-4,-1.95)(4.25,4.1)
%\psgrid
\def\tobo{\psplot[plotpoints=2000,linewidth=1.25pt]{-3}{4}{2.71828 x exp x dup mul 1 add div}}
\psplot[plotpoints=2000,linewidth=1.25pt]{-3}{4}{2.71828 x exp x dup mul 1 add div}
\rput(-0.6,0.6){\tobo}
\psline(-3,0)(-3.6,0.6)\psline(4,3.2)(3.4,3.8)
\pscustom[fillstyle=solid,fillcolor=lightgray]
{\psplot[plotpoints=2000,linewidth=1.25pt,linecolor=red]{-3}{4}{2.71828 x exp x dup mul 1 add div}
\psline(4,3.2)(4,0)(-2,0)}
\rput(0,-1){Vue de profil du toboggan}
\end{pspicture}
\end{minipage}

	\begin{enumerate}
		\item \textit{D'après le graphique ci-dessus, le toboggan semble-t-il assurer de bonnes sensations ? Argumenter}.
		
		La fonction $f$ semble convexe sur $[-4;-0,3]$ puis concave sur $[-0,3;1]$ puis convexe sur $[1;4]$.

		Elle change donc de convexité et $\mathcal{C}_f$ semble présenter deux points d'inflexion. Le tobbogan seble donc assurer de bonnes senstations.

		\item On admet que la fonction $f''$, dérivée seconde de la fonction $f$, a pour expression pour tout réel $x$ de l'intervalle $[-3~;~4]$ :

		\[f''(x) = \dfrac{p(x)(x - 1)\text{e}^x}{\left(1 + x^2\right)^3}\]
 avec pour tout réel $x\in[-3;4]$, $p(x)= x^3 - 3x^2 + 5x + 1$,  et on admet également que le signe de $p(x)$ est donné par le tableau suivant:

\begin{center}
	\begin{tikzpicture}
		\tkzTabInit{$x$ / 1 , $p(x)$ / 1}{$-3$, $\alpha$, $4$}
		\tkzTabLine{, -, z, +, }
	\end{tikzpicture}
	\end{center}
avec $\alpha\approx -0,2$ à $10^{-1}$ près.



\textit{En utilisant l'expression précédente de $f''$, répondre à la question : \og le toboggan assure-t-il de bonnes sensations ? \fg. Justifier.}

On étudie la convexité de $f$, donc on étudie le signe de $f''(x)$. D'après l'expression admise, $f''(x)$ est du signe de $p(x)(x-1)$. On en déduit le tableau :




\begin{center}
	\begin{tikzpicture}
		\tkzTabInit{$x$ / 1 , $p(x)$ / 1, $x-1$ / 1, $f''(x)$ / 1.5}{$-3$, $\alpha$ ,$1$, $4$}
		\tkzTabLine{, -, z, +, , +}
		\tkzTabLine{, -, , -, z, +}
		\tkzTabLine{, +, z, -, z, +}
	 \end{tikzpicture}
	\end{center}

	$f''$ s'annule deux fois en changeant de signe (ou $f$ change deux fois de convexité), donc $\mathcal{C}_f$ présente deux points d'inflexion.

	Le tobbogan assure donc de bonnes sensations.

	\end{enumerate}
	
	
	
	\item (Bonus) : Montrer que pour tout réel $x$ de l'intervalle $[-3~;~4]$ :

		\[f''(x) = \dfrac{(x^3-3x^2+5x+1)(x - 1)\text{e}^x}{\left(1 + x^2\right)^3}\]

\end{enumerate}

\bigskip


\end{exerciceinterro}

\pagebreak



\begin{exerciceinterro}{\hspace{2cm}/5 points}{}


En mai 2020, une entreprise fait le choix de développer le télétravail afin de s'inscrire dans une démarche écoresponsable.

Elle propose alors à ses \np{5000}~collaborateurs en France de choisir entre le télétravail et le travail au sein des locaux de l'entreprise.

\medskip

Afin d'évaluer l'impact de cette mesure sur son personnel, les dirigeants de l'entreprise sont parvenus à modéliser le nombre de collaborateurs satisfaits par ce dispositif à l'aide de la suite 
$\left(u_n\right)$ définie par $u_0 = 1$ et, pour tout entier naturel $n$,

\[u_{n+1}  = \dfrac{5u_n + 4}{u_n + 2}\]

où $u_n$ désigne le nombre de milliers de collaborateurs satisfaits par cette nouvelle mesure au bout de $n$ mois après le mois de mai 2020.

\medskip

\begin{enumerate}
\item Démontrer que la fonction $f$ définie pour tout $x \in  [0~;~ +\infty[$ par $f(x) = \dfrac{5x+4}{x+2}$ est 
strictement croissante sur $[0~;~+ \infty[$.
		\item Démontrer par récurrence que pour tout entier naturel 
		\[0 \leqslant u_n \leqslant u_{n+1}  \leqslant  5.\]
		
		\item Justifier que la suite $\left(u_n\right)$ est convergente.

\item En déduire la limite de la suite $\left(u_n\right)$ et l'interpréter dans le contexte de l'exercice.
\end{enumerate}

\end{exerciceinterro}


\begin{exerciceinterro}{\hspace{2cm}/4 points}{}


Dans tout cet exercice, les probabilités seront arrondies, si nécessaire, à $10^{-3}$.

D'après une étude, les utilisateurs réguliers de transports en commun représentent 17\,\% de la population française. 

Parmi ces utilisateurs réguliers, 32\,\% sont des jeunes âgés de 18 à 24 ans. (Source : TNS-Sofres)

\bigskip

\textbf{Partie A} :

\medskip
 
On interroge une personne au hasard et on note :

\setlength\parindent{9mm}
\begin{itemize}
\item $R$ l'évènement : \og La personne interrogée utilise régulièrement les transports en commun \fg.
\item $J$ l'évènement : \og La personne interrogée est âgée de 18 à 24 ans \fg.
\end{itemize}
\setlength\parindent{0mm}

\medskip

\begin{enumerate}
\item Représentez la situation à l'aide de cet arbre pondéré, que vous recopierez sur votre copie, en y reportant les données de l'énoncé.
\begin{center}
\pstree[treemode=R,nodesepB=3pt,levelsep=2.8cm]{\TR{}}
{\pstree{\TR{$R$~~}}
		{\TR{$J$}
		\TR{$\overline{J}$}
 		}
\pstree{\TR{$\overline{R}$~~}}
		{\TR{$J$}
		\TR{$\overline{J}$}
		}
}
\end{center}

\item Calculer la probabilité $P(R \cap J)$.
\item D'après cette même étude, les jeunes de 18 à 24 ans représentent 11\,\% de la population française.

Montrer que la probabilité que la personne interrogée soit un jeune de 18 à 24 ans n'utilisant pas régulièrement les transports en commun est 0,056 à $10^{-3}$ près.
\item En déduire la proportion de jeunes de 18 à 24 ans parmi les utilisateurs non réguliers des transports en commun.
\end{enumerate}

\bigskip

\textbf{Partie B} :

\medskip

Lors d'un recensement sur la population française, un recenseur interroge au hasard 50 personnes en une journée sur leur pratique des transports en commun. 

La population française est suffisamment importante pour assimiler ce recensement à un tirage avec remise.

Soit $X$ la variable aléatoire dénombrant les personnes utilisant régulièrement les transports en commun parmi les 50 personnes interrogées.

\medskip

\begin{enumerate}
\item Déterminer, en justifiant, la loi de $X$ et préciser ses paramètres.
\item Calculer $P(X = 5)$ et interpréter le résultat.
\item Le recenseur indique qu'il y a moins de 5\,\% de chance pour que, parmi les $50$ personnes interrogées, $13$ personnes ou plus utilisent régulièrement les transports en commun.

Cette affirmation est-elle vraie ? Justifier votre réponse.
\item Quel est le nombre moyen de personnes utilisant régulièrement les transports en commun parmi les $50$ personnes interrogées ?
\end{enumerate}


\end{exerciceinterro}






\end{document}
        