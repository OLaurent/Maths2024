\documentclass[a4paper]{article}

\usepackage[T1]{fontenc}
\usepackage[french]{babel}
\usepackage{ProfLycee}
\usepackage{math-vh}

\usepackage{pst-tree,pst-plot,pst-text,pst-func,pst-math,pst-bspline,pstricks-add,pst-arrow}

% *** Réglage des headers ***
\fancyhead[L]{TMATHS4 --- Lycée Victor Hugo}
\fancyhead[R]{O. Laurent --- 3 février 2024}
% *** Réglage des headers ***

\begin{document}

\begin{center}
  {\scshape\LARGE Interrogation \No5\par}
  \vspace{0.5cm}
\end{center}

\NomPrenom{}

\vspace{1cm}



\begin{exerciceinterro}{\hspace{1cm}/6 points}{}
  On considère les points $A(-1;1;2)$, $B(1;0;-1)$, $C(0;3;1)$ et $D(-8;2;-3)$.
  \begin{enumerate}
    \item Démontrer que les points $A$, $B$ et $C$ définissent bien un plan.
    \item Démontrer que le vecteur $\overrightarrow{AD}$ est un vecteur normal au plan $(ABC)$.
    \item Calculer $\overrightarrow{BA}\cdot\overrightarrow{BC}$.
    \item Calculer les longueurs $BA$ et $BC$.
    \item En déduire la mesure de l'angler $\widehat{ABC}$.  Arrondir à $0,1$ degré près.
  \end{enumerate}
  \end{exerciceinterro}

  \bigskip
    
\begin{exerciceinterro}{\hspace{1cm}/4 points}{}
$ABCD$ est un cube.
\begin{center}
  \psset{unit=0.5cm}
  \begin{pspicture}(-0.5,0)(9.5,7.5)

    \pspolygon[fillcolor=lightgray, fillstyle=solid, linestyle=dashed](8.9,7.2)(4.2,7.2)(6.8,0.5)
    \psline[fillcolor=lightgray, fillstyle=solid,](8.9,7.2)(6.8,0.5)


  % Cube ABCDEFGH
  \psframe(2,0.5)(6.8,5.2)%ABFE
  \psline(6.8,0.5)(8.9,2.4)(8.9,7.2)(6.8,5.2)%BCGF
  \psline(8.9,7.2)(4.2,7.2)(2,5.1)%GHE
  \psline[linestyle=dashed](2,0.5)(4.2,2.4)(4.2,7.2)%ADH
  \psline[linestyle=dashed](8.9,2.4)(4.2,2.4)
  \psdots(2,0.5)(6.8,0.5)(8.9,2.4)(4.2,2.4)%ABCD
  \uput[dl](2,0.5){E} \uput[d](6.8,0.5){F} \uput[r](8.9,2.4){G} \uput[ul](4.2,2.4){H}
  \psdots(2,5.2)(6.8,5.2)(8.9,7.2)(4.2,7.2) %EFGH
  \uput[l](2,5.2){A} \uput[u](6.8,5.2){B} \uput[ur](8.9,7.2){C} \uput[u](4.2,7.2){D}

    
  \psline[linecolor=red]{}(8.9,2.4)(6.8,5.2)
\end{pspicture}
\end{center}
\begin{enumerate}
  \item Justifier que la droite $(BG)$ est orthogonale à la droite $(FC)$.
  \item Démontrer que la droite $(DC)$ est orthogonale au plan $(BCG)$.
  \item En déduire que la droite $(BG)$ est orthogonale au plan $(DCF)$.
\end{enumerate}
  \end{exerciceinterro}

\end{document}
        