\documentclass[a4paper]{article}
\usepackage{silence}
%Disable all warnings issued by latex starting with "You have..."
\WarningFilter{latex}{You have requested package}

\usepackage[T1]{fontenc}
\usepackage[french]{babel}
\usepackage{ProfLycee}
\useproflyclib{ecritures}
\usepackage{math-vh}
\usepackage{yhmath}
% *** Réglage des headers ***
\fancyhead[L]{1MATHS9 --- Lycée Victor Hugo}
\fancyhead[R]{O. Laurent --- Année 2023--2024}
% *** Réglage des headers ***

\begin{document}

\begin{center}
  {\scshape\LARGE Chapitre 6 --- Suites I\par}
  \vspace{0.5cm}
\end{center}
\begin{center}
  {\LARGE Exercices sur les sens de variation \par}
  \vspace{0.5cm}
\end{center}

\begin{exerciceinterro}{}{}
Déterminer le sens de variation des suites définies ci-dessous:
\begin{enumerate}
    \item 			$\begin{cases}
        u_0 &=-5\\
        u_{n+1} &=u_n+n+3\\
\end{cases}$
\item 			$\begin{cases}
    v_0 &=1\\
    v_{n+1} &=v_n(1-v_n)\\
\end{cases}$

   
\end{enumerate}
\end{exerciceinterro}


\begin{exerciceinterro}{}{}
Déterminer le sens de variation des suites suivantes définies de façon explicite:
\begin{enumerate}
    \item $u_n=2-4n$
    \item $v_n=n^2+3$
    \item $w_n=n^2+2n$
    \item $t_n=2^n+3^n$
\end{enumerate}

\end{exerciceinterro}
\begin{exerciceinterro}{}{}
    Déterminer le sens de variation de la suite $(u_n)$ définie pour tout entier naturel $n$ par
    $$u_n=n^2-8n+2$$
\end{exerciceinterro}

\begin{exerciceinterro}{}{}
Soit la suite $(u_n)$ définie par:
$$\begin{cases}
    u_0 &=1\\
    u_{n+1} &=\dfrac{u_n}{u_n+1}\\
\end{cases}$$
\begin{enumerate}
    \item A l'aide de la calculatrice, conjecturer le sens de variation de cette suite.
    \item On admet que tous les termes de cette suite sont positifs. Justifier alors la conjecture obtenue à la question précédente.
\end{enumerate}
\end{exerciceinterro}

\begin{exerciceinterro}{}{}
On considère la suite $(u_n)$ définie par son premier terme et la relation de récurrence:
$$u_{n+1}=u_n^2-2u_n-3$$
\begin{enumerate}
    \item \textbf{Cas où $u_0=3$}
    \begin{enumerate}
        \item Calculer $u_1$, $u_2$ et $u_3$.
        \item Peut-on dire que, $\forall n \in \N, u_n >0$? Justifier la réponse.
        \item Afficher les valeurs des vingt premiers termes sur la calculatrice.
        
        Quel résultat la calculatrice affiche pour $u_{10}$? Expliquer.
    \end{enumerate}
    \item \textbf{Cas où $u_0=-2$}
    
    Reprendre les mêmes questions.
\end{enumerate}

\end{exerciceinterro}

\pagebreak
\begin{exerciceinterro}{}{}
    On s'intéresse à la suite défine par son premier terme et la relation de récurrence:
    $$u_{n+1}=-\dfrac{1}{2}u_n^2+u_n+1$$
    Sur les graphiques ci-dessous, on a représenté la droite $\mathcal{D}$ d'équation $y=x$ ainsi que la courbe $\mathcal{C}_f$ représentant la fonction $f:x\longmapsto -\dfrac{1}{2}x^2+x+1$.
	
    \begin{enumerate}
        \item On choisit dans un premier temps $u_0=-1$. Placer sur l'axe des abscisses les 4 premiers termes de la suite $(u_n)$.
        \begin{center}
            \begin{tikzpicture}[x=2cm,y=2cm,%unités
            xmin=-2,xmax=2,xgrille=1,xgrilles=1, %axe Ox
            ymin=-2,ymax=2,ygrille=1,ygrilles=1] %axe Oy
                \GrilleTikz
                \AxexTikz[Police=\small]{-2,-1,1}
                \AxeyTikz[Police=\small]{-2,-1,1}
                \AxesTikz
    
    
                 \def\f{-0.5*\x*\x+\x+1}
                %toile
               % \ToileRecurrence[Fct={\f},No=0,Uno=-1, Nb=10,DecalLabel=4pt]
                %éléments supplémentaires
                \draw[very thick,blue,domain=-1.7:2,samples=250] plot (\x,{\f}) ;
                \draw[very thick,CouleurVertForet,domain=-2:2,samples=2] plot (\x,\x) ;
    
    
            \end{tikzpicture}
        \end{center}
        \item On choisit maintenant $u_0=0,1$. Placer sur l'axe des abscisses les 5 premiers termes de la suite $(u_n)$.
           \begin{center}
            \begin{tikzpicture}[x=5cm,y=5cm,%unités
            xmin=0,xmax=2,xgrille=0.1,xgrilles=1, %axe Ox
            ymin=0,ymax=2,ygrille=0.1,ygrilles=1] %axe Oy
                \GrilleTikz
                \AxexTikz[Police=\small]{1,2}
                \AxeyTikz[Police=\small]{1,2}
                \AxesTikz
    
    
                 \def\f{-0.5*\x*\x+\x+1}
                %toile
               % \ToileRecurrence[Fct={\f},No=0,Uno=0.1, Nb=10,DecalLabel=4pt]
                %éléments supplémentaires
                \draw[very thick,blue,domain=0:2,samples=250] plot (\x,{\f}) ;
                \draw[very thick,CouleurVertForet,domain=0:2,samples=2] plot (\x,\x) ;
    
    
            \end{tikzpicture}
        \end{center}

        Que constate-t-on? Retrouver ce résultat par le calcul.

    \end{enumerate}

	
\end{exerciceinterro}
\end{document}
        