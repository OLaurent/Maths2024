\documentclass[a4paper]{article}

\usepackage{ProfLycee}
\useproflyclib{ecritures}
\usepackage{math-vh}
% *** Réglage des headers ***
\fancyhead[L]{2NDE --- Lycée Victor Hugo}
\fancyhead[R]{O. Laurent --- Année 2024--2025}
% *** Réglage des headers ***



\begin{document}

\begin{center}
  {\scshape\LARGE Chapitre 10 --- Vecteurs II\par}
\end{center}


\begin{savoirfaire}{}{}
    \begin{itemize}[label=$\square$]
        \item Représenter un vecteur dont on connait les coordonnées.
        \item Lire les coordonnées d’un vecteur.
        \item Calculer les coordonnées d’une somme de vecteurs, d’un produit d’un vecteur par un nombre réel.
        \item Caractériser alignement et parallélisme par la colinéarité de vecteurs.
        \item Résoudre des problèmes en utilisant la représentation la plus adaptée des vecteurs.
    \end{itemize}
\end{savoirfaire}


\section{Coordonnées d'un vecteur}
\begin{definition}{}{}
  Soit $O$ un point et deux vecteurs $\Vecteur{i}$ et $\Vecteur{j}$ dont les directions sont 
  perpendiculaires et dont les normes sont égales à 1.

  On dit que $(\Vecteur{i},\Vecteur{j})$ est une \textbf{base orthonormée} du plan 
  et que $(O;\Vecteur{i},\Vecteur{j})$ est un \textbf{repère orthonormé du plan}.
\end{definition}

\begin{definition}{}{}
  Dans une base $(\Vecteur{i},\Vecteur{j})$, soit $\Vecteur{u}$ un vecteur. Il existe un unique couple $(x;y)$ tel que $\Vecteur{u}=x\Vecteur{i}+y\Vecteur{j}$.


  $x$ et $y$ sont les coordonnées de $\Vecteur{u}$ dans la base $(\Vecteur{i},\Vecteur{j})$, 
  notées $\CoordVecPl{x}{y}$.
\end{definition}

\begin{example}{}{}
 
  \vspace{0.5cm}
 
 \begin{tabular}{cc}
    \begin{minipage}{0.5\textwidth}
      \begin{center}
        \begin{tikzpicture}[scale=0.8]
        \coordinate (O) at (1,1);
        \coordinate (I) at (2,1);
        \coordinate (J) at (1,2);
        \coordinate (A) at (2,2);
        \coordinate (B) at (5,4);
        
        \draw [very thin, gray] (0,0) grid[step=1] (8,5);
        
        \draw[very thick, ->] (1,0) -- (1,5);
        \draw[very thick, ->] (0,1) -- (8,1);
        
       % \draw (A) node[cross=2pt,black]{};
       % \node[above left] at ($(A)$){$A$};
       % \draw (B) node[cross=2pt,black]{};
       % \node[above right] at ($(B)$){$B$};
        \draw (O) node[cross=2pt,black]{};
        \node[below left] at ($(O)$){$O$};
       % \draw (I) node[cross=2pt,black]{};
       % \node[above left] at ($(I)$){$I$};
       % \draw (J) node[cross=2pt,black]{};
       % \node[above left] at ($(J)$){$J$};
        
        
       \draw[very thick, color=ForestGreen, ->] ($(O)$) -- ($(I)$) node[midway, below] {$\Vecteur{i}$};
       \draw[very thick, color=ForestGreen, ->] ($(O)$) -- ($(J)$) node[midway, left] {$\Vecteur{j}$};
    
    
        \draw[very thick, color=red, ->] ($(A)$) -- ($(B)$) node[midway, above] {$\Vecteur{u}$};
        \draw[very thick, color=black, dashed, ->] ($(A)$) -- (5,2) node[midway, above, fill=white] {$3\Vecteur{i}$};
        \draw[very thick, color=black, dashed, ->] (5,2) -- ($(B)$)   node[midway, right, fill=white] {$2\Vecteur{j}$};
        \end{tikzpicture} 
      \end{center}     
    \end{minipage}&
    \begin{minipage}{0.5\textwidth}
      Dans la base $(\Vecteur{i},\Vecteur{j})$, $\Vecteur{u}$ a ainsi pour coordonnées $\CoordVecPl{3}{2}$.


      \SimpleQRCode{https://www.youtube.com/watch?v=AdpyGuma3Q8}{2cm}
      \SimpleQRCode{https://www.youtube.com/watch?v=8PyiMHtp1fE}{2cm}
    \end{minipage}
  \end{tabular}


\end{example}


\begin{proprieteadm}{Coordonnées d'un vecteur}{}
  Dans un repère, si deux points $A$ et $B$ ont pour coordonnées respectives $(x_A;y_A)$ et $(x_B;y_B)$, le vecteur $\Vecteur{AB}$ a pour coordonnées $\CoordVecPl{x_B-x_A}{y_B-y_A}$.
\end{proprieteadm}
\begin{example}{}{}

    \begin{center}
        \begin{tabular}{cc}
    \begin{minipage}{0.5\textwidth}
        \begin{tikzpicture}[scale=1]
            \coordinate (O) at (1,1);
            \coordinate (I) at (2,1);
            \coordinate (J) at (1,2);
            \coordinate (A) at (2,4);
            \coordinate (B) at (5,2);
            
            \draw [very thin, gray] (0,0) grid[step=1] (8,5);
            
            \draw[very thick, ->] (1,0) -- (1,5);
            \draw[very thick, ->] (0,1) -- (8,1);
            
            \draw (A) node[cross=2pt,black]{};
            \node[above left] at ($(A)$){$A$};
            \draw (B) node[cross=2pt,black]{};
            \node[above right] at ($(B)$){$B$};
            \draw (O) node[cross=2pt,black]{};
            \node[below left] at ($(O)$){$O$};
            \draw (I) node[cross=2pt,black]{};
            % \draw (I) node[cross=2pt,black]{};
            % \node[above left] at ($(I)$){$I$};
            % \draw (J) node[cross=2pt,black]{};
            % \node[above left] at ($(J)$){$J$};
             
             
            \draw[very thick, color=ForestGreen, ->] ($(O)$) -- ($(I)$) node[midway, below] {$\Vecteur{i}$};
            \draw[very thick, color=ForestGreen, ->] ($(O)$) -- ($(J)$) node[midway, left] {$\Vecteur{j}$};
            
            
            \draw[very thick, color=red, ->] ($(A)$) -- ($(B)$);
            \draw[very thick, color=black, dashed, ->] ($(A)$) -- (5,4) node[midway, above, fill=white] {$x_B-x_A=3$};
            \draw[very thick, color=black, dashed, ->] (5,4) -- ($(B)$)   node[midway, right, fill=white] {$y_B-y_A=-2$};
            \end{tikzpicture} 
            
    \end{minipage}&
    \begin{minipage}{0.5\textwidth}
        Sur le graphique ci-contre, on a : $A(1;3)$ et $B(4;1)$.
        
        Les coordonnées du vecteur $\Vecteur{AB}$ sont donc $\CoordVecPl{4-1}{1-3}$ soit $\Vecteur{AB}\CoordVecPl{3}{-2}$
    \end{minipage}
\end{tabular}
  \end{center}
  \vspace{0.5cm}
\end{example}
\vspace{0.5cm}

\begin{proprieteadm}{Caractérisation analytique de l'égalité de deux vecteurs}{}
  Deux vecteurs sont égaux si et seulement si ils ont les mêmes coordonnées dans un repère du plan.

  Autrement dit, si les vecteurs $\Vecteur{u}\CoordVecPl{x}{y}$ et $\Vecteur{v}\CoordVecPl{x'}{y'}$ sont égaux, alors 
  $x=x'$ et $y=y'$.
\end{proprieteadm}

\begin{exercices}{}{}
  80--82 p.139, 83--89 p.140
\end{exercices}

\begin{methode}{Montrer qu'un quadrilatère est un parallélogramme}{}
  Soient $A(2;3)$, $B(5;1)$, $C(3;-2)$ et $D(0;0)$.
  
  Montrer que $ABCD$ est un parallélogramme.

  \begin{itemize}
    \item On calcule les coordonnées de $\Vecteur{AB}$ et de $\Vecteur{DC}$.
    \vspace{2cm}
    \item On vérifie qu'elles sont égales. 
    \vspace{2cm}
  \end{itemize}



\end{methode}


\begin{definition}{}{}
  Dans un repère orthonormé $(O,I,J)$, soit $\Vecteur{u}$ un vecteur de coordonnées $\CoordVecPl{x}{y}$.\\
  La norme du vecteur $\Vecteur{u}$, notée $ \left\lVert \Vecteur{u} \right\rVert$, est 
  $$ \left\lVert \Vecteur{u} \right\rVert = \sqrt{x^2+y^2} $$ 
\end{definition}

\begin{definition}{}{}
  Dans un repère orthonormé $(O,I,J)$, soient $A\CoordPtPl{x_A}{y_A}$ et $B\CoordPtPl{x_B}{y_B}$. Alors :
  $$ \left\lVert \Vecteur{AB} \right\rVert = \sqrt{(x_B-x_A)^2+(y_B-y_A)^2} $$ 
\end{definition}


\begin{propriete}{Coordonnées de la somme de deux vecteurs}{}
  Dans un repère, si $\Vecteur{u}\CoordVecPl{x}{y}$  et $\Vecteur{v}\CoordVecPl{x'}{y'}$, alors $\Vecteur{u}+\Vecteur{v}\CoordVecPl{x+x'}{y+y'}$
\end{propriete}

\begin{demonstration}{}{}
  \begin{tabular}{p{12cm} p{4cm}}
    \begin{minipage}{12cm}
      Dans un repère d'origine $O$, la translation de vecteur $\Vecteur{u}(x;y)$ associe au point $O$ le point $M(x;y)$. La translation de vecteur $\Vecteur{v}(x';y')$ associe au point $M$ le point $N$. Alors, $\Vecteur{u}+\Vecteur{v}=\Vecteur{ON}$.
      
      Cherchons les coordonnées de $N$ :

      Les coordonnées de $\Vecteur{MN}$ sont $(x_N-x;y_N-y)$. Or, $\Vecteur{MN}=\Vecteur{v}$, c'est-à-dire $x_N-x=x'$ et $y_N-y=y'$.
      
      On en déduit que $N$ a pour coordonnées $(x+x';y+y')$, d'où $\Vecteur{u}+\Vecteur{v}\CoordVecPl{x+x'}{y+y'}$.
    \end{minipage}
    &\begin{minipage}{4cm}
      \begin{tikzpicture}[scale=1]
      %Parallelogram ABCD and its diagonals are drawn.
      \coordinate (O) at (0,0);
      \coordinate (M) at (1,-1);
      \coordinate (N) at (2,2);
      
      
      
      \draw (M) node[cross=2pt,black]{};
      \draw (N) node[cross=2pt,black]{};
      \draw (O) node[cross=2pt,black]{};
      \draw[very thick,color=red,->,shorten >=0.5mm] (O) -- (M);
      \draw[very thick,color=blue,->,shorten >=0.5mm] (M) -- (N);
      \draw[very thick,color=green,->,shorten >=0.5mm] (O) -- (N);
      
      
      \draw[very thick,color=black,->] (-1,0) -- (2.5,0);
      \draw[very thick,color=black,->] (0,-1.5) -- (0,2.5);
      
      \node[below left] at ($(M)$){$M$};
      \node[above right] at ($(N)$){$N$};
      \node[below left] at ($(O)$){$O$};
      \node[below left] at ($(O)! 0.5 !(M)$){$\Vecteur{u}$};
      \node[above left] at ($(M)! 0.5 !(N)$){$\Vecteur{v}$};
      \node[above left] at ($(O)! 0.5 !(N)$){$\Vecteur{u}+\Vecteur{v}$};
      
      
      
      \draw[very thick, color=black, dashed, ->] ($(M)$) -- (2,-1) node[midway, below, fill=white] {$x'$};
      \draw[very thick, color=black, dashed, ->] (2,-1) -- ($(N)$)   node[midway, right, fill=white] {$y'$};
      
      
      \draw[very thick, color=black, dashed] ($(M)$) -- (1,0) node[ above, fill=white] {$x$};
      \draw[very thick, color=black, dashed] ($(M)$) -- (0,-1) node[ left, fill=white] {$y$};
      
      \end{tikzpicture}
    \end{minipage}
    \\
  \end{tabular}
  
  

\end{demonstration}
\vspace{0.5cm}

\begin{example}{}{}
  Soient $\Vecteur{u}\CoordVecPl{-3}{5}$  et $\Vecteur{v}\CoordVecPl{10}{-8}$, alors $\Vecteur{u}+\Vecteur{v}\CoordVecPl{7}{-3}$
\end{example}


\begin{definition}{Déterminant de deux vecteurs}{}
Soit $\left(\Vecteur{i},\Vecteur{j}\right)$ une base orthonormée et deux vecteurs $\Vecteur{u}\CoordVecPl{x}{y}$ et $\Vecteur{v}\CoordVecPl{x'}{y'}$.

On appelle \textbf{déterminant} de $\Vecteur{u}$ et $\Vecteur{v}$ dans la base $\left(\Vecteur{i},\Vecteur{j}\right)$ le nombre $det\left(\Vecteur{u},\Vecteur{v}\right)=xy'-yx'$, noté également 
$\begin{vmatrix} 
  x & x' \\ 
  y & y' 
  \end{vmatrix} $
\end{definition}

\begin{propriete}{}{}
  Soit $\left(\Vecteur{i},\Vecteur{j}\right)$ une base orthonormée et deux vecteurs $\Vecteur{u}\CoordVecPl{x}{y}$ et $\Vecteur{v}\CoordVecPl{x'}{y'}$.

  $\Vecteur{u}$ et $\Vecteur{v}$ sont colinéaires si et seulement si $det\left(\Vecteur{u},\Vecteur{v}\right)=0$.
\end{propriete}

\begin{demonstrationp}{}{}
  \begin{itemize}
    \item Supposons que $\Vecteur{u}$ et $\Vecteur{v}$ sont colinéaires.
    \begin{itemize}
      \item Si l'un des deux vecteurs est nul (par exemple $\Vecteur{u}$), alors $det\left(\Vecteur{u},\Vecteur{v}\right)=0\times y'-0 \times x'=0$
      \item Sinon, il existe un nombre $k$ tel que $\Vecteur{v}=k\Vecteur{x}$, soit $x'=kx$ et $y'=ky$. Alors $det\left(\Vecteur{u},\Vecteur{v}\right)=xy'-yx'=x\times ky-y\times kx=kxy-kxy=0$.
    \end{itemize}

    \item Réciproquement, supposons que $xy'-yx'=0$.
    
    On a alors $xy'=yx'$. 
    \begin{itemize}
      \item Si l'un des vecteurs est nul, alors il est nécessairement colinéaire à l'autre.
      \item Si les deux vecteurs sont non nuls, alors $\Vecteur{u}$ a au moins une coordonnée non nulle, par exemple $x$, donc $x\neq0$.
        On pose alors $k=\dfrac{x'}{x}$, et on obtient que $xy'=yx' \iff y'=\dfrac{yx'}{x} \iff y'=ky$, car $x\neq 0$.

        Par conséquent, $\Vecteur{v}=k\Vecteur{u}$, et les vecteurs $\Vecteur{u}$ et $\Vecteur{v}$ sont colinéaires.
    \end{itemize}
  \end{itemize}

\end{demonstrationp}

\begin{example}{}{}

  Soient dans une base $\Vecteur{u}\CoordVecPl{12}{-26}$ et $\Vecteur{v}\CoordVecPl{35}{-72}$.

  Alors $\begin{vmatrix} 
    12 & 35 \\ 
    -26 & -72 
    \end{vmatrix} = 12\times(-72)-(-26)\times 35=46\neq 0$.

    $det\left(\Vecteur{u},\Vecteur{v}\right)\neq 0$,
      donc les deux vecteurs ne sont pas colinéaires.
\end{example}

\begin{exercices}{}{}
  108--110 p.141, 112--117 p.141, 118--120 p.142
\end{exercices}


  
\end{document}