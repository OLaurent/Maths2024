\documentclass[a4paper]{article}

\usepackage{ProfLycee}
\useproflyclib{ecritures}
\useproflyclib{pythontex}
\usepackage{math-vh}
\usepackage{tabularx}
% *** Réglage des headers ***
\fancyhead[L]{2NDE --- Lycée Victor Hugo}
\fancyhead[R]{O. Laurent --- Année 2024--2025}
% *** Réglage des headers ***



\begin{document}

\begin{center}
  {\scshape\LARGE Chapitre 10 --- Vecteurs II --- Exercices corrigés\par}
\end{center}

\begin{exercice}{(*)}{}

  \begin{tabular}{cc}
    \begin{minipage}{7cm}
      \begin{center}
        \begin{tikzpicture}[scale=0.5,y=1cm,xmin=-4,xmax=7,ymin=-1,ymax=6,
          xgrille=1,xgrilles=1,ygrille=1,ygrilles=1]
          \GrilleTikz %grille
          \AxesTikz %axes
          \AxeyTikz[AffGrad=false]{}
    
          \draw[very thick, ->, Red!50!white] (0,0) -- (1,0) node[midway, below] {$\Vecteur{i}$};
          \draw[very thick, ->, Red!50!white] (0,0) -- (0,1) node[midway, left] {$\Vecteur{j}$};
    
    
          \draw[very thick, ->, ForestGreen] (-3,1) -- (-1,5) node[midway, above left] {$\Vecteur{v}$};
          \draw[very thick, ->, Blue] (1,2) -- (6,5) node[midway, above] {$\Vecteur{u}$};
          \draw[very thick, ->, Red] (3,0) -- (5,0) node[midway, above] {$\Vecteur{w}$};
        \end{tikzpicture}
      \end{center}   
    \end{minipage}&
    \begin{minipage}{8cm}
      Par lecture graphique, déterminer les coordonnées dans la base $(\Vecteur{i}, \Vecteur{j})$ des vecteurs $\Vecteur{u}$, $\Vecteur{v}$ et $\Vecteur{w}$ ci-dessous.
\begin{center}

  \begin{tabularx}{.95\linewidth}{X X X }
    \textbf{1)} $\Vecteur{u}\CoordVecPl{5}{3}$ & \textbf{2)} $\Vecteur{v}\CoordVecPl{2}{4}$ & 
    \textbf{3)}  $\Vecteur{w}\CoordVecPl{5}{-3}$\\
    \end{tabularx}
\end{center}
    \end{minipage}
  \end{tabular}

\end{exercice}

\begin{exercice}{(*)}{}
  Soient les points $A(x_A;y_A)$ et $B(x_B;y_B)$. Dans chaque cas, déterminer les coordonnées du vecteur $\Vecteur{AB}$.
  \begin{center}
    \begin{tabularx}{.95\linewidth}{X X X }
      \textbf{1)}$A(2;3)$ et $B(5;7)$: $\Vecteur{AB}\CoordVecPl{3}{4}$ & \textbf{2)} $A(-1;4)$ et $B(3;1)$: $\Vecteur{AB}\CoordVecPl{4}{-3}$ & 
      \textbf{3)} $A(-4;-9)$ et $B(3;-8)$: $\Vecteur{AB}\CoordVecPl{7}{1}$\\
    \end{tabularx}
  \end{center}
\end{exercice}

\begin{exercice}{(*)}{}

  \begin{tabular}{cc}
    \begin{minipage}{7cm}
      \begin{center}
        \begin{tikzpicture}[scale=0.5,y=1cm,xmin=-4,xmax=7,ymin=-1,ymax=6,
          xgrille=1,xgrilles=1,ygrille=1,ygrilles=1]
          \GrilleTikz %grille
          \AxesTikz %axes
          \AxeyTikz[AffGrad=false]{}
    
          \draw[very thick, ->, Red!50!white] (0,0) -- (1,0) node[midway, below] {$\Vecteur{i}$};
          \draw[very thick, ->, Red!50!white] (0,0) -- (0,1) node[midway, left] {$\Vecteur{j}$};
    
    
          \draw[very thick, ->, ForestGreen] (-1,5) -- (3,1) node[midway, above right] {$\Vecteur{v}$};
          \draw[very thick, ->, Blue] (1,2) -- (6,3) node[midway, above] {$\Vecteur{u}$};
          \draw[very thick, ->, Red] (0,1) -- (-3,5) node[midway, above right] {$\Vecteur{w}$};

          \draw (1,2) node{$\bullet$} node[above left]{$A$};
          \draw (-1,5) node{$\bullet$} node[above right]{$B$};
          \draw (0,1) node{$\bullet$} node[below]{$C$};
        \end{tikzpicture}
      \end{center}   
    \end{minipage}&
    \begin{minipage}{10cm}
      On considère les vecteurs $\Vecteur{u}\CoordVecPl{5}{1}$, $\Vecteur{v}\CoordVecPl{4}{-4}$ et $\Vecteur{w}\CoordVecPl{-3}{4}$.

  Représenter ces vecteurs en choisissant comme origine respectivement les points $A(1;2)$, $B(-1;5)$ et $C(0;1)$.
    \end{minipage}
    
  \end{tabular}
 
 
\end{exercice}

\begin{exercice}{(*)}{}
  Soient les points $E(3;6)$, $H(-5;8)$ et $K(-1;7)$.
  \begin{enumerate}
    \item \textit{Montrer que les vecteurs $\Vecteur{EK}$ et $\Vecteur{KH}$ sont égaux.}
  
    On calcule les coordonnées de ces vecteurs: $\Vecteur{EK}\CoordVecPl{-1-3}{7-6}$ soit $\Vecteur{EK}\CoordVecPl{-4}{-1}$
    et $\Vecteur{KH}\CoordVecPl{-5-(-1)}{8-7}$ soit $\Vecteur{KH}\CoordVecPl{-4}{1}$.

    On a bien $\Vecteur{EK}=\Vecteur{KH}$.

    \item \textit{Que peut-on en déduire?}
    
    On en déduit d'une part que les points $E$, $K$ et $H$ sont alignés et d'autre part que $EK=KH$.

    $K$ est le milieu du segment $[EH]$.
  \end{enumerate}
\end{exercice}

\begin{exercice}{(**)}{}
 Soient les points $A(2;5)$, $B(-1;3)$, $C(4;-1)$ et $D(7;1)$.
 \begin{enumerate}
  \item \textit{Montrer que le quadrilatère $ABCD$ est un parallélogramme.}

  On calcule les coordonnées des vecteurs $\Vecteur{AB}$ et $\Vecteur{DC}$: $\Vecteur{AB}\CoordVecPl{-1-2}{3-5}$ soit $\Vecteur{AB}\CoordVecPl{-3}{-2}$ et $\Vecteur{DC}\CoordVecPl{4-7}{-1-1}$ soit $\Vecteur{DC}\CoordVecPl{-3}{-2}$.

  On a bien $\Vecteur{AB}=\Vecteur{DC}$, donc $ABCD$ est un parallélogramme.

  \item \textit{Calculer les coordonnées du point $G$ tel que $ABGC$ soit un parallélogramme.}
  
  On cherche les coordonnées du point $G(x_G;y_G)$ tel que $\Vecteur{AB}=\Vecteur{CG}$.

  On a $\Vecteur{CG}\CoordVecPl{x_G-4}{y_G+1}$. On a donc le système suivant:
  $\begin{dcases}
    x_G-4=-3\\
    y_G+1=-2
  \end{dcases} \iff \begin{dcases}
    x_G=1\\
    y_G=-3
  \end{dcases}$. Donc $G(1;-3)$.


 \end{enumerate}
\end{exercice}

\pagebreak

\begin{exercice}{(**)}{}
Soient les points $A(-4;2)$, $B(1;2)$, $C(-1;6)$, $D(0;-1)$ et $E(5;-1)$ dans le repère orthonormé $\RepereOij$.
  \begin{enumerate}
    \item \begin{enumerate}
      \item \textit{Montrer que le quadrilatère $ABED$ est un parallélogramme.}
    
      On calcule les coordonnées des vecteurs $\Vecteur{AB}$ et $\Vecteur{DE}$: $\Vecteur{AB}\CoordVecPl{1-(-4)}{2-2}$ soit $\Vecteur{AB}\CoordVecPl{5}{0}$ et $\Vecteur{DE}\CoordVecPl{5-0}{-1-(-1)}$ soit $\Vecteur{DE}\CoordVecPl{5}{0}$.

      On a bien $\Vecteur{AB}=\Vecteur{DE}$, donc $ABED$ est un parallélogramme.

      \item \textit{Calculer les longueurs $AB$ et $EB$. Que peut-on en déduire?}
      
      On calcule $AB=\sqrt{(1-(-4))^2+(2-2)^2}=\sqrt{5^2}=5$ et $EB=\sqrt{(5-1)^2+(-1-2)^2}=\sqrt{4^2+3^2}=\sqrt{16+9}=\sqrt{25}=5$.

      On a alors $AB=EB$, donc $ABED$ est un losange.

    \end{enumerate}
    \item \textit{Calculer les coordonnées du point $G$ tel que $ABCG$ soit un parallélogramme.}
 
    On cherche les coordonnées du point $G(x_G;y_G)$ tel que $\Vecteur{AB}=\Vecteur{GC}$.

    On a $\Vecteur{GC}\CoordVecPl{x_G-(-1)}{y_G-6}$. On a donc le système suivant:
    $\begin{dcases}
      x_G+1=5\\
      y_G-6=0
    \end{dcases} \iff \begin{dcases}
      x_G=4\\
      y_G=6
    \end{dcases}$. Donc $G(4;6)$.

  
    \item \textit{Le parallélogramme $ABCG$ est-il un losange? Justifier.}
    
    On calcule $AB=\sqrt{(1-(-4))^2+(2-2)^2}=\sqrt{5^2}=5$ et $BC=\sqrt{(-1-1)^2+(6-2)^2}=\sqrt{(-2)^2+4^2}=\sqrt{4+16}=\sqrt{20}$.

    On a alors $AB\neq BC$, donc $ABCG$ n'est pas un losange.
  \end{enumerate}
\end{exercice}

\begin{exercice}{(**)}{}
  Soient les points $A(1;2)$, $B(3;-2)$ et les vecteurs $\Vecteur{u}\CoordVecPl{2}{5}$ et $\Vecteur{v}\CoordVecPl{1}{-2}$.

  \begin{enumerate}
    \item \textit{Calculer les coordonnées du vecteur $\Vecteur{u}+\Vecteur{v}$.}
    
    On a $\Vecteur{u}+\Vecteur{v}\CoordVecPl{2+1}{5+(-2)}$ soit $\Vecteur{u}+\Vecteur{v}\CoordVecPl{3}{3}$.

    \item \textit{Calculer les coordonnées des points $E$ et $F$ tels que $\Vecteur{AE}=\Vecteur{u}+\Vecteur{v}$ et $\Vecteur{BF}=\Vecteur{u}+\Vecteur{v}$.}

    On a $\Vecteur{AE}\CoordVecPl{x_E-1}{y_E-2}$ et $\Vecteur{BF}\CoordVecPl{x_F-3}{y_F-(-2)}$.

    On a donc le système suivant:
    $\begin{dcases}
      x_E-1=3\\
      y_E-2=3
    \end{dcases} \iff \begin{dcases}
      x_E=4\\
      y_E=5
    \end{dcases}$ et $\begin{dcases}
      x_F-3=3\\
      y_F+2=3
    \end{dcases} \iff \begin{dcases}
      x_F=6\\
      y_F=1
    \end{dcases}$.

    Donc $E(4;5)$ et $F(6;1)$.
  \end{enumerate}
\end{exercice}

\begin{exercice}{(**)}{}
  Soient les points $A(-3;2)$, $B(-1;3)$, $C(1;1)$ et $D(9;-1)$.

  Les points $M$ et $N$ sont définis par $\begin{dcases}
    \Vecteur{AM}=\Vecteur{AB}+\Vecteur{CD} \\
    \Vecteur{BN}=\Vecteur{BA}+\Vecteur{BC}  
  \end{dcases}$.

  \begin{enumerate}
    \item \textit{Calculer les coordonnées des points $M$ et $N$.}
    
    On a $\Vecteur{AB}\CoordVecPl{-1-(-3)}{3-2}$ soit $\Vecteur{AB}\CoordVecPl{2}{1}$ et $\Vecteur{CD}\CoordVecPl{9-1}{-1-1}$ soit $\Vecteur{CD}\CoordVecPl{8}{-2}$.
   
    On a donc $\Vecteur{AB}+\Vecteur{CD}\CoordVecPl{2+8}{1+(-2)}$ soit $\Vecteur{AB}+\Vecteur{CD}\CoordVecPl{10}{-1}$.

    De même, on a $\Vecteur{BA}\CoordVecPl{-3-(-1)}{2-3}$ soit $\Vecteur{BA}\CoordVecPl{-2}{-1}$ et $\Vecteur{BC}\CoordVecPl{1-(-1)}{1-3}$ soit $\Vecteur{BC}\CoordVecPl{2}{-2}$.
   
    On a donc $\Vecteur{BA}+\Vecteur{BC}\CoordVecPl{-2+2}{-1+(-2)}$ soit $\Vecteur{BA}+\Vecteur{BC}\CoordVecPl{0}{-3}$.


    On cherche les coordonnées du point $M(x_M;y_M)$ tel que $\Vecteur{AM}=\Vecteur{AB}+\Vecteur{CD}$.

    On a $\Vecteur{AM}\CoordVecPl{x_M-(-3)}{y_M-2}$ soit $\Vecteur{AM}\CoordVecPl{x_M+3}{y_M-2}$. On a donc le système suivant:
    $\begin{dcases}
      x_M+3=10\\
      y_M-2=-1
    \end{dcases} \iff \begin{dcases}
      x_M=7\\
      y_M=1
    \end{dcases}$. Donc $M(7;1)$.

    De même, on cherche les coordonnées du point $N(x_N;y_N)$ tel que $\Vecteur{BN}=\Vecteur{BA}+\Vecteur{BC}$.

    On a $\Vecteur{BN}\CoordVecPl{x_N-(-1)}{y_N-3}$ soit $\Vecteur{BN}\CoordVecPl{x_N+1}{y_N-3}$. On a donc le système suivant:
    $\begin{dcases}
      x_N+1=0\\
      y_N-3=-3
    \end{dcases} \iff \begin{dcases}
      x_N=-1\\
      y_N=0
    \end{dcases}$. Donc $N(-1;0)$.

    \item \textit{Montrer que le quadrilatère $ANDM$ est un parallélogramme.}
    
    On calcule les coordonnées des vecteurs $\Vecteur{AN}$ et $\Vecteur{MD}$:
    
    $\Vecteur{AN}\CoordVecPl{-1-(-3)}{0-2}$ soit $\Vecteur{AN}\CoordVecPl{2}{-2}$ et $\Vecteur{MD}\CoordVecPl{9-7}{-1-1}$ soit $\Vecteur{MD}\CoordVecPl{2}{-2}$.

    On a bien $\Vecteur{AN}=\Vecteur{MD}$, donc $ANDM$ est un parallélogramme.

  \end{enumerate}
\end{exercice}

\begin{exercice}{(**)}{}
 Soient les points $A(2;-1)$, $B(3;7)$, $C(-5;1)$ et $K(11;13)$.
 \begin{enumerate}
  \item \textit{Calculer les coordonnées des vecteurs $\Vecteur{AB}$ et $\Vecteur{BC}$, puis celles du vecteur $-\Vecteur{AB}+2\Vecteur{BC}$.}
  
  On a $\Vecteur{AB}\CoordVecPl{3-2}{7-(-1)}$ soit $\Vecteur{AB}\CoordVecPl{1}{8}$, $\Vecteur{BC}\CoordVecPl{-5-3}{1-7}$ soit $\Vecteur{BC}\CoordVecPl{-8}{-6}$.

  On a donc $-\Vecteur{AB}+2\Vecteur{BC}\CoordVecPl{-1+2\times(-8)}{-8+2\times(-6)}$ soit $-\Vecteur{AB}+2\Vecteur{BC}\CoordVecPl{-17}{-20}$.
  \item \textit{Calculer les coordonnées du point $L$ défini par $\Vecteur{BL}=-\Vecteur{AB}+2\Vecteur{BC}$.}
  
  On cherche les coordonnées du point $L(x_L;y_L)$ tel que $\Vecteur{BL}=-\Vecteur{AB}+2\Vecteur{BC}$.

  On a $\Vecteur{BL}\CoordVecPl{x_L-3}{y_L-7}$ et $-\Vecteur{AB}+2\Vecteur{BC}\CoordVecPl{-17}{-20}$.

  On a donc le système suivant:
  $\begin{dcases}
    x_L-3=-17\\
    y_L-7=-20
  \end{dcases} \iff \begin{dcases}
    x_L=-14\\
    y_L=-13
  \end{dcases}$. Donc $L(-14;-13)$.

  \item \textit{Montrer que le quadrilatère $CKAL$ est un parallélogramme}.
  
  On a $\Vecteur{CK}\CoordVecPl{11-(-5)}{13-1}$ soit $\Vecteur{CK}\CoordVecPl{16}{12}$ et $\Vecteur{LA}\CoordVecPl{2-(-14)}{(-1)-(-13)}$ soit $\Vecteur{LA}\CoordVecPl{16}{12}$.

  On a donc $\Vecteur{CK}=\Vecteur{LA}$, donc $CKAL$ est un parallélogramme.
\end{enumerate}
\end{exercice}

\begin{exercice}{(*)}{}
Dans chaque cas, déterminer si les vecteurs $\Vecteur{u}$ et $\Vecteur{v}$ sont colinéaires.
\begin{center}
  \begin{tabularx}{.95\linewidth}{X X}
  \textbf{1)}  $\Vecteur{u}\CoordVecPl{24}{6}$ et $\Vecteur{v}\CoordVecPl{8}{2}$ & \textbf{2)} $\Vecteur{u}\CoordVecPl{20}{-10}$ et $\Vecteur{v}\CoordVecPl{-15}{5}$\\
  \end{tabularx}
\end{center}

\begin{enumerate}
  \item On calcule le déterminant des vecteurs $\Vecteur{u}$ et $\Vecteur{v}$: $det(\Vecteur{u}, \Vecteur{v})=\begin{vmatrix}
    24 & 8\\
    6 & 2
  \end{vmatrix}=24\times 2-6\times 8=48-48=0$.
  
  Les vecteurs $\Vecteur{u}$ et $\Vecteur{v}$ sont donc colinéaires car le déterminant est nul.

  \item On calcule le déterminant des vecteurs $\Vecteur{u}$ et $\Vecteur{v}$: $det(\Vecteur{u}, \Vecteur{v})=\begin{vmatrix}
    20 & -15\\
    -10 & 5 
  \end{vmatrix}=20\times 5-(-10)\times(-15)=100-150=-50$.

  Les vecteurs $\Vecteur{u}$ et $\Vecteur{v}$ ne sont donc pas colinéaires car le déterminant est non nul.

\end{enumerate}
\end{exercice}

\begin{exercice}{(**)}{}
  Dans chaque cas, déterminer la valeur du réel $k$ tel que les vecteurs $\Vecteur{u}$ et $\Vecteur{v}$ soient colinéaires.
   \begin{center}
    \begin{tabularx}{.95\linewidth}{X X}
    \textbf{1)} $\Vecteur{u}\CoordVecPl{-3}{4}$ et $\Vecteur{v}\CoordVecPl{k}{2}$ & \textbf{2)} $\Vecteur{u}\CoordVecPl{5}{1}$ et $\Vecteur{v}\CoordVecPl{6}{3k}$\\
    \end{tabularx}
  \end{center}

  \begin{enumerate}
    \item On cherche $k$ tel que les vecteurs $\Vecteur{u}$ et $\Vecteur{v}$ soient colinéaires.
    
    On a $det(\Vecteur{u}, \Vecteur{v})=\begin{vmatrix}
      -3 & k\\
      4 & 2 
    \end{vmatrix}=-3\times 2-4\times k=-6-4k$.

    On cherche donc le réel $k$ tel que $-6-4k=0 \iff -4k=6 \iff k=-\dfrac{6}{4}=-\dfrac{3}{2}$. Les vecteurs $\Vecteur{u}$ et $\Vecteur{v}$ sont donc colinéaires pour $k=-\dfrac{3}{2}$.

    \item On cherche $k$ tel que les vecteurs $\Vecteur{u}$ et $\Vecteur{v}$ soient colinéaires.

    On a $det(\Vecteur{u}, \Vecteur{v})=\begin{vmatrix}
      5 & 6\\
      1 & 3k
    \end{vmatrix}=5\times 3k-1\times 6=15k-6$.

    On cherche donc le réel $k$ tel que $15k-6=0 \iff 15k=6 \iff k=\dfrac{6}{15}=\dfrac{2}{5}$. Les vecteurs $\Vecteur{u}$ et $\Vecteur{v}$ sont donc colinéaires pour $k=\dfrac{2}{5}$.
  \end{enumerate}
  \end{exercice}

  \begin{exercice}{(**)}{}
    Dans chaque cas, déterminer si les droites $(AB)$ et $(CD)$ sont parallèles.
    \begin{enumerate}
      \item \textit{$A(1;1)$, $B(3;11)$, $C(0;-1)$ et $D(-1;-7)$}
      
      On calcule les coordonnées des vecteurs $\Vecteur{AB}$ et $\Vecteur{CD}$: $\Vecteur{AB}\CoordVecPl{3-1}{11-1}$ soit $\Vecteur{AB}\CoordVecPl{2}{10}$ et $\Vecteur{CD}\CoordVecPl{-1-0}{(-7)-(-1)}$ soit $\Vecteur{CD}\CoordVecPl{-1}{-6}$.

      Les droites $(AB)$ et $(CD)$ sont parallèles si et seulement si les vecteurs $\Vecteur{AB}$ et $\Vecteur{CD}$ sont colinéaires.

      On a $det(\Vecteur{AB}, \Vecteur{CD})=\begin{vmatrix}
        2 & -1\\
        10 & -6
      \end{vmatrix}=2\times(-6)-10\times(-1)=-12+10=-2$. Comme le déterminant est non nul, les droites $(AB)$ et $(CD)$ ne sont pas parallèles.


      
      \item \textit{$A(3;10)$, $B(0;-5)$, $C(1;-20)$ et $D(10;25)$}
      
      On calcule les coordonnées des vecteurs $\Vecteur{AB}$ et $\Vecteur{CD}$: $\Vecteur{AB}\CoordVecPl{0-3}{-5-10}$ soit $\Vecteur{AB}\CoordVecPl{-3}{-15}$ et $\Vecteur{CD}\CoordVecPl{10-1}{25-(-20)}$ soit $\Vecteur{CD}\CoordVecPl{9}{45}$.

      Les droites $(AB)$ et $(CD)$ sont parallèles si et seulement si les vecteurs $\Vecteur{AB}$ et $\Vecteur{CD}$ sont colinéaires.

      On a $det(\Vecteur{AB}, \Vecteur{CD})=\begin{vmatrix}
        -3 & 9\\
        -15 & 45
      \end{vmatrix}=-3\times 45-(-15)\times 9=-135+135=0$. Comme le déterminant est nul, les droites $(AB)$ et $(CD)$ sont parallèles.

    \end{enumerate}
    \end{exercice}

    \begin{exercice}{(**)}{}
      Dans chaque cas, dire si les points $A$, $B$ et $C$ sont alignés ou non.
      \begin{enumerate}
        \item $A(1;3)$, $B(-1;2)$ et $C(2;3)$
        
        On calcule les coordonnées des vecteurs $\Vecteur{AB}$ et $\Vecteur{AC}$: $\Vecteur{AB}\CoordVecPl{-1-1}{2-3}$ soit $\Vecteur{AB}\CoordVecPl{-2}{-1}$ et $\Vecteur{AC}\CoordVecPl{2-1}{3-3}$ soit $\Vecteur{AC}\CoordVecPl{1}{0}$.

        Les points $A$, $B$ et $C$ sont alignés si et seulement si les vecteurs $\Vecteur{AB}$ et $\Vecteur{AC}$ sont colinéaires.

        On a $det(\Vecteur{AB}, \Vecteur{AC})=\begin{vmatrix}
          -2 & 1\\
          -1 & 0
        \end{vmatrix}=-2\times 0-(-1)\times 1=0+1=1$. Comme le déterminant est non nul, les points $A$, $B$ et $C$ ne sont pas alignés.

        \item $A(\sqrt{2};3)$, $B(0;1)$ et $C(2\sqrt{2};5)$
        
        On calcule les coordonnées des vecteurs $\Vecteur{AB}$ et $\Vecteur{AC}$: $\Vecteur{AB}\CoordVecPl{0-\sqrt{2}}{1-3}$ soit $\Vecteur{AB}\CoordVecPl{-\sqrt{2}}{-2}$ et $\Vecteur{AC}\CoordVecPl{2\sqrt{2}-\sqrt{2}}{5-3}$ soit $\Vecteur{AC}\CoordVecPl{\sqrt{2}}{2}$.

        Les points $A$, $B$ et $C$ sont alignés si et seulement si les vecteurs $\Vecteur{AB}$ et $\Vecteur{AC}$ sont colinéaires.

        On a $det(\Vecteur{AB}, \Vecteur{AC})=\begin{vmatrix}
          -\sqrt{2} & \sqrt{2}\\
          -2 & 2
        \end{vmatrix}=-\sqrt{2}\times 2-(-2)\times \sqrt{2}=-2\sqrt{2}+2\sqrt{2}=0$. Comme le déterminant est nul, les points $A$, $B$ et $C$ sont alignés.
         
      \end{enumerate}
      \end{exercice}

  \begin{exercice}{(**)}{}


  \begin{tabular}{cc}
    \begin{minipage}{7cm}
     
  \begin{center}
    \begin{tikzpicture}[scale=0.5,y=1cm,xmin=-1,xmax=10,ymin=-4,ymax=5,
      xgrille=1,xgrilles=1,ygrille=1,ygrilles=1]
      \GrilleTikz %grille
      \AxesTikz %axes
      \AxeyTikz[AffGrad=false]{}

      \draw (1,3) node{$\bullet$} node[above left]{$A$};
      \draw (9,-1) node{$\bullet$} node[above right]{$B$};
      \draw (4,-3) node{$\bullet$} node[below]{$C$};
      \draw[Red] (5,1) node{$\bullet$} node[below]{$D$};
      \draw[Red] (7,0) node{$\bullet$} node[below]{$E$};
      

      \draw[very thick, ->, Red!50!white] (0,0) -- (1,0) node[midway, below] {$\Vecteur{i}$};
      \draw[very thick, ->, Red!50!white] (0,0) -- (0,1) node[midway, left] {$\Vecteur{j}$};

      \draw[Blue] (1,3) -- (4,-3) -- (9,-1) -- cycle;

%      \draw[very thick, ->, ForestGreen] (-3,1) -- (-1,5) node[midway, above left] {$\Vecteur{v}$};
 %     \draw[very thick, ->, Blue] (1,2) -- (6,5) node[midway, above] {$\Vecteur{u}$};
  %    \draw[very thick, ->, Red] (0,0) -- (5,3) node[midway, above] {$\Vecteur{w}$};
    \end{tikzpicture}
  \end{center}   
    \end{minipage}&
    \begin{minipage}{8cm}
     
    On considère les points $A(1;3)$, $B(9;-1)$, $C(4;-3)$ dans un repère $\RepereOij$.
    \begin{enumerate}
      \item Calculer les coordonnées du milieu $D$ du segment $[AB]$ et celles du milieu $E$ du segment $[DB]$.
      \item Calculer les coordonnées du point $S$ défini par $\Vecteur{AS}=\dfrac{2}{3}\Vecteur{AC}$.
      \item Les droites $(EC)$ et $(DS)$ sont-elles parallèles? Justifier.
    \end{enumerate}


    \end{minipage}
  \end{tabular}

  \begin{enumerate}
    \item On calcule les coordonnées du milieu $D$ du segment $[AB]$ et celles du milieu $E$ du segment $[DB]$.
    
    On a $D(\dfrac{1+9}{2};\dfrac{3+(-1)}{2})$ soit $D(5;1)$ et $E(\dfrac{9+5}{2};\dfrac{-1+1}{2})$ soit $E(7;0)$.

    \item On calcule les coordonnées du point $S$ défini par $\Vecteur{AS}=\dfrac{2}{3}\Vecteur{AC}$.
    
    On a $\Vecteur{AC}\CoordVecPl{4-1}{-3-3}$ soit $\Vecteur{AC}\CoordVecPl{3}{-6}$.

    On a $\Vecteur{AS}\CoordVecPl{x_S-1}{y_S-3}$. On a donc le système suivant:
    $\begin{dcases}
      x_S-1=\dfrac{2}{3}\times 3\\
      y_S-3=\dfrac{2}{3}\times (-6)
    \end{dcases} \iff \begin{dcases}
      x_S-1=2\\
      y_S-3=-4
    \end{dcases} \iff \begin{dcases}
      x_S=3\\
      y_S=-1
    \end{dcases}$. Donc $S(3;-1)$.

    \item Les droites $(EC)$ et $(DS)$ sont parallèles si et seulement si les vecteurs $\Vecteur{EC}$ et $\Vecteur{DS}$ sont colinéaires.
    
    On a $\Vecteur{EC}\CoordVecPl{4-7}{-3-0}$ soit $\Vecteur{EC}\CoordVecPl{-3}{-3}$ et $\Vecteur{DS}\CoordVecPl{3-5}{-1-1}$ soit $\Vecteur{DS}\CoordVecPl{-2}{-2}$.

    On a $det(\Vecteur{EC}, \Vecteur{DS})=\begin{vmatrix}
      -3 & -2\\
      -3 & -2
    \end{vmatrix}=-3\times(-2)-(-3)\times(-2)=6-6=0$. Comme le déterminant est nul, les droites $(EC)$ et $(DS)$ sont parallèles.

    
  \end{enumerate}
    \end{exercice}


    \begin{exercice}{}{}
    Compléter ce script en Python permettant de déterminer si deux vecteurs sont colinéaires.
      \begin{center}
        \begin{CodePythontex}[Largeur=0.75\linewidth, Centre, Lignes=false]{}
          def vecteurs_colineaires(u, v):
          if u[0] * v[1] == u[1] * v[0]:
            return True
          else:
            return False
          \end{CodePythontex}
      \end{center}


      Ce code en Python définit une fonction \texttt{vecteurs\_colineaires} qui vérifie si deux vecteurs sont colinéaires. 

\begin{enumerate}
    \item \textbf{Définition de la fonction} : La fonction \texttt{vecteurs\_colineaires} prend deux vecteurs \texttt{u} et \texttt{v} comme arguments. Un vecteur est représenté par une liste de deux éléments, par exemple, \texttt{[x, y]}.
    \item \textbf{Condition de colinéarité} : La fonction vérifie si les vecteurs \texttt{u} et \texttt{v} sont colinéaires en utilisant la condition suivante : \texttt{u[0] * v[1] == u[1] * v[0]}. Cette condition est dérivée de la propriété des vecteurs colinéaires, qui stipule que les vecteurs sont colinéaires si et seulement si le produit croisé de leurs composantes est nul.
    \item \textbf{Retourne le résultat} : Si la condition est vraie, la fonction retourne \texttt{True}, sinon elle retourne \texttt{False}.
\end{enumerate}

    \end{exercice}
\end{document}