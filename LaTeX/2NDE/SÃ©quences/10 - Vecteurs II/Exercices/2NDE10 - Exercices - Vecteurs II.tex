\documentclass[a4paper]{article}

\usepackage{ProfLycee}
\useproflyclib{ecritures}
\useproflyclib{pythontex}
\usepackage{math-vh}
\usepackage{tabularx}
% *** Réglage des headers ***
\fancyhead[L]{2NDE --- Lycée Victor Hugo}
\fancyhead[R]{O. Laurent --- Année 2024--2025}
% *** Réglage des headers ***



\begin{document}

\begin{center}
  {\scshape\LARGE Chapitre 10 --- Vecteurs II --- Exercices\par}
\end{center}

\begin{exercice}{(*)}{}

  \begin{tabular}{cc}
    \begin{minipage}{7cm}
      \begin{center}
        \begin{tikzpicture}[scale=0.5,y=1cm,xmin=-4,xmax=7,ymin=-1,ymax=6,
          xgrille=1,xgrilles=1,ygrille=1,ygrilles=1]
          \GrilleTikz %grille
          \AxesTikz %axes
          \AxeyTikz[AffGrad=false]{}
    
          \draw[very thick, ->, Red!50!white] (0,0) -- (1,0) node[midway, below] {$\Vecteur{i}$};
          \draw[very thick, ->, Red!50!white] (0,0) -- (0,1) node[midway, left] {$\Vecteur{j}$};
    
    
          \draw[very thick, ->, ForestGreen] (-3,1) -- (-1,5) node[midway, above left] {$\Vecteur{v}$};
          \draw[very thick, ->, Blue] (1,2) -- (6,5) node[midway, above] {$\Vecteur{u}$};
          \draw[very thick, ->, Red] (0,0) -- (5,3) node[midway, above] {$\Vecteur{w}$};
        \end{tikzpicture}
      \end{center}   
    \end{minipage}&
    \begin{minipage}{8cm}
      Par lecture graphique, déterminer les coordonnées dans la base $(\Vecteur{i}, \Vecteur{j})$ des vecteurs $\Vecteur{u}$, $\Vecteur{v}$ et $\Vecteur{w}$ ci-dessous.
\begin{center}

  \begin{tabularx}{.95\linewidth}{X X X }
    \textbf{1)} $\Vecteur{u}\CoordVecPl{....}{....}$ & \textbf{2)} $\Vecteur{v}\CoordVecPl{....}{....}$ & 
    \textbf{3)}  $\Vecteur{w}\CoordVecPl{....}{....}$\\
    \end{tabularx}
\end{center}
    \end{minipage}
  \end{tabular}

\end{exercice}

\begin{exercice}{(*)}{}
  Soient les points $A(x_A;y_A)$ et $B(x_B;y_B)$. Dans chaque cas, déterminer les coordonnées du vecteur $\Vecteur{AB}$.
  \begin{center}
    \begin{tabularx}{.95\linewidth}{X X X }
      \textbf{1)}$A(2;3)$ et $B(5;7)$: $\Vecteur{AB}\CoordVecPl{....}{....}$ & \textbf{2)} $A(-1;4)$ et $B(3;1)$: $\Vecteur{AB}\CoordVecPl{....}{....}$ & 
      \textbf{3)} $A(-4;-9)$ et $B(3;-8)$: $\Vecteur{AB}\CoordVecPl{....}{....}$\\
    \end{tabularx}
  \end{center}
\end{exercice}

\begin{exercice}{(*)}{}

  \begin{tabular}{cc}
    \begin{minipage}{7cm}
      \begin{center}
        \begin{tikzpicture}[scale=0.5,y=1cm,xmin=-4,xmax=7,ymin=-1,ymax=6,
          xgrille=1,xgrilles=1,ygrille=1,ygrilles=1]
          \GrilleTikz %grille
          \AxesTikz %axes
          \AxeyTikz[AffGrad=false]{}
    
          \draw[very thick, ->, Red!50!white] (0,0) -- (1,0) node[midway, below] {$\Vecteur{i}$};
          \draw[very thick, ->, Red!50!white] (0,0) -- (0,1) node[midway, left] {$\Vecteur{j}$};
    
    
          %\draw[very thick, ->, ForestGreen] (-3,1) -- (-1,5) node[midway, above left] {$\Vecteur{v}$};
          %\draw[very thick, ->, Blue] (1,2) -- (6,5) node[midway, above] {$\Vecteur{u}$};
          %\draw[very thick, ->, Red] (0,0) -- (5,3) node[midway, above] {$\Vecteur{w}$};
        \end{tikzpicture}
      \end{center}   
    \end{minipage}&
    \begin{minipage}{10cm}
      On considère les vecteurs $\Vecteur{u}\CoordVecPl{5}{1}$, $\Vecteur{v}\CoordVecPl{2}{-3}$ et $\Vecteur{w}\CoordVecPl{-3}{4}$.

  Représenter ces vecteurs en choisissant comme origine respectivement les points $A(1;2)$, $B(-2;4)$ et $C(0;1)$.
    \end{minipage}
    
  \end{tabular}
 
 
\end{exercice}

\begin{exercice}{(*)}{}
  Soient les points $E(3;6)$, $H(-5;8)$ et $K(-1;7)$.
  \begin{enumerate}
    \item Montrer que les vecteurs $\Vecteur{EK}$ et $\Vecteur{KH}$ sont égaux.
    \item Que peut-on en déduire?
  \end{enumerate}
\end{exercice}

\begin{exercice}{(**)}{}
 Soient les points $A(2;5)$, $B(-1;3)$, $C(4;-1)$ et $D(7;1)$.
 \begin{enumerate}
  \item Montrer que le quadrilatère $ABCD$ est un parallélogramme.
  \item Calculer les coordonnées du point $G$ tel que $ABGC$ soit un parallélogramme.
 \end{enumerate}
\end{exercice}

\begin{exercice}{(**)}{}
Soient les points $A(-4;2)$, $B(1;2)$, $C(-1;6)$, $D(0;-1)$ et $E(5;-1)$ dans le repère orthonormé $\RepereOij$.
  \begin{enumerate}
    \item \begin{enumerate}
      \item Montrer que le quadrilatère $ABED$ est un parallélogramme.
      \item Calculer les longueurs $AB$ et $EB$. Que peut-on en déduire?
    \end{enumerate}
    \item Calculer les coordonnées du point $G$ tel que $ABCG$ soit un parallélogramme.
    \item Le parallélogramme $ABCG$ est-il un losange? Justifier.
  \end{enumerate}
\end{exercice}

\begin{exercice}{(**)}{}
  Soient les points $A(1;2)$, $B(3;-2)$ et les vecteurs $\Vecteur{u}\CoordVecPl{2}{5}$ et $\Vecteur{v}\CoordVecPl{1}{-2}$.

  \begin{enumerate}
    \item Calculer les coordonnées du vecteur $\Vecteur{u}+\Vecteur{v}$.
    \item Calculer les coordonnées des points $E$ et $F$ tels que $\Vecteur{AE}=\Vecteur{u}+\Vecteur{v}$ et $\Vecteur{BF}=\Vecteur{u}+\Vecteur{v}$.
  \end{enumerate}
\end{exercice}

\begin{exercice}{(**)}{}
  Soient les points $A(-3;2)$, $B(-1;3)$, $C(1;1)$ et $D(9;-1)$.

  Les points $M$ et $N$ sont définis par $\begin{dcases}
    \Vecteur{AM}=\Vecteur{AB}+\Vecteur{CD} \\
    \Vecteur{BN}=\Vecteur{BA}+\Vecteur{BC}  
  \end{dcases}$.

  \begin{enumerate}
    \item Calculer les coordonnées des points $M$ et $N$.
    \item Montrer que le quadrilatère $ANDM$ est un parallélogramme.
  \end{enumerate}
\end{exercice}

\begin{exercice}{(**)}{}
 Soient les points $A(2;-1)$, $B(3;7)$, $C(-5;1)$ et $K(11;13)$.
 \begin{enumerate}
  \item Calculer les coordonnées des vecteurs $\Vecteur{AB}$ et $\Vecteur{BC}$, puis celles du vecteur $-\Vecteur{AB}+2\Vecteur{BC}$.
  \item Calculer les coordonnées du point $L$ défini par $\Vecteur{BL}=-\Vecteur{AB}+2\Vecteur{BC}$.
  \item Montrer que le quadrilatère $CKAL$ est un parallélogramme.
\end{enumerate}
\end{exercice}

\begin{exercice}{(*)}{}
Dans chaque cas, déterminer si les vecteurs $\Vecteur{u}$ et $\Vecteur{v}$ sont colinéaires.
\begin{center}
  \begin{tabularx}{.95\linewidth}{X X}
  \textbf{1)}  $\Vecteur{u}\CoordVecPl{24}{6}$ et $\Vecteur{v}\CoordVecPl{8}{2}$ & \textbf{2)} $\Vecteur{u}\CoordVecPl{20}{-10}$ et $\Vecteur{v}\CoordVecPl{-15}{5}$\\
  \end{tabularx}
\end{center}
\end{exercice}

\begin{exercice}{(**)}{}
  Dans chaque cas, déterminer la valeur du réel $k$ tel que les vecteurs $\Vecteur{u}$ et $\Vecteur{v}$ soient colinéaires.
   \begin{center}
    \begin{tabularx}{.95\linewidth}{X X}
    \textbf{1)} $\Vecteur{u}\CoordVecPl{-3}{4}$ et $\Vecteur{v}\CoordVecPl{k}{2}$ & \textbf{2)} $\Vecteur{u}\CoordVecPl{5}{1}$ et $\Vecteur{v}\CoordVecPl{6}{3k}$\\
    \end{tabularx}
  \end{center}

  \end{exercice}

  \begin{exercice}{(**)}{}
    Dans chaque cas, déterminer si les droites $(AB)$ et $(CD)$ sont parallèles.
    \begin{enumerate}
      \item $A(1;1)$, $B(3;11)$, $C(0;-1)$ et $D(-1;-7)$
      \item $A(3;10)$, $B(0;-5)$, $C(1;-20)$ et $D(10;25)$
    \end{enumerate}
    \end{exercice}

    \begin{exercice}{(**)}{}
      Dans chaque cas, dire si les points $A$, $B$ et $C$ sont alignés ou non.
      \begin{enumerate}
        \item $A(1;3)$, $B(-1;2)$ et $C(2;3)$
        \item $A(\sqrt{2};3)$, $B(0;1)$ et $C(2\sqrt{2};1)$
      \end{enumerate}
      \end{exercice}

  \begin{exercice}{(**)}{}


  \begin{tabular}{cc}
    \begin{minipage}{7cm}
     
  \begin{center}
    \begin{tikzpicture}[scale=0.5,y=1cm,xmin=-1,xmax=10,ymin=-4,ymax=5,
      xgrille=1,xgrilles=1,ygrille=1,ygrilles=1]
      \GrilleTikz %grille
      \AxesTikz %axes
      \AxeyTikz[AffGrad=false]{}

      \draw (1,3) node{$\bullet$} node[above left]{$A$};
      \draw (9,-1) node{$\bullet$} node[above right]{$B$};
      \draw (4,-3) node{$\bullet$} node[below]{$C$};
      \draw[Red] (5,1) node{$\bullet$} node[below]{$D$};
      \draw[Red] (7,0) node{$\bullet$} node[below]{$E$};
      

      \draw[very thick, ->, Red!50!white] (0,0) -- (1,0) node[midway, below] {$\Vecteur{i}$};
      \draw[very thick, ->, Red!50!white] (0,0) -- (0,1) node[midway, left] {$\Vecteur{j}$};

      \draw[Blue] (1,3) -- (4,-3) -- (9,-1) -- cycle;

%      \draw[very thick, ->, ForestGreen] (-3,1) -- (-1,5) node[midway, above left] {$\Vecteur{v}$};
 %     \draw[very thick, ->, Blue] (1,2) -- (6,5) node[midway, above] {$\Vecteur{u}$};
  %    \draw[very thick, ->, Red] (0,0) -- (5,3) node[midway, above] {$\Vecteur{w}$};
    \end{tikzpicture}
  \end{center}   
    \end{minipage}&
    \begin{minipage}{8cm}
     
    On considère les points $A(1;3)$, $B(9;-1)$, $C(4;-3)$ dans un repère $\RepereOij$.
    \begin{enumerate}
      \item Calculer les coordonnées du milieu $D$ du segment $[AB]$ et celles du milieu $E$ du segment $[DB]$.
      \item Calculer les coordonnées du point $S$ défini par $\Vecteur{AS}=\dfrac{2}{3}\Vecteur{AC}$.
      \item Les droites $(EC)$ et $(DS)$ sont-elles parallèles? Justifier.
    \end{enumerate}


    \end{minipage}
  \end{tabular}
    \end{exercice}


    \begin{exercice}{}{}
    Compléter ce script en Python permettant de déterminer si deux vecteurs sont colinéaires.
      \begin{center}
        \begin{CodePythontexAlt}[Largeur=0.75\linewidth, Centre, Lignes=false]{}
          def colineaires(u, v):
          if u[0]*v[1]==u[1]*v[0]:
            return True
          else:
            return Falseaaa
          \end{CodePythontexAlt}
      \end{center}
    \end{exercice}
\end{document}